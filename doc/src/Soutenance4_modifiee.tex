\documentclass[a4paper,11pt]{amsart}

\usepackage[francais]{babel}
%\usepackage[pdftex]{graphicx}
%\uspackage{pstricks}
\usepackage{graphicx}
\newtheorem{remarque}{Remarque} \newcommand{\AAA}{{\mathcal A}}
\newcommand{\BB}{{\mathcal B}} \newcommand{\CC}{{\mathcal C}}
\newcommand{\DD}{{\mathcal D}} \newcommand{\EE}{{\mathcal E}}
\newcommand{\FF}{{\mathcal F}} \newcommand{\GG}{{\mathcal G}}
\newcommand{\HH}{{\mathcal H}} \newcommand{\II}{{\mathcal I}}
\newcommand{\JJ}{{\mathcal J}} \newcommand{\KK}{{\mathcal K}}
\newcommand{\LL}{{\mathcal L}} \newcommand{\MM}{{\mathcal M}}
\newcommand{\NN}{{\mathcal N}} \newcommand{\OO}{{\mathcal O}}
\newcommand{\PP}{{\mathcal P}} \newcommand{\QQ}{{\mathcal Q}}
\newcommand{\RR}{{\mathcal R}} \newcommand{\SSS}{{\mathcal S}}
\newcommand{\TT}{{\mathcal T}} \newcommand{\UU}{{\mathcal U}}
\newcommand{\VV}{{\mathcal V}} \newcommand{\WW}{{\mathcal W}}
\newcommand{\XX}{{\mathcal X}} \newcommand{\ZZ}{{\mathcal Z}}
\newcommand{\bbR}{{\mathbb R}} \newcommand{\bbD}{{\mathbb D}}
\newcommand{\bbO}{{\mathbb O}} \newcommand{\bbS}{{\mathbb S}}
\newcommand{\bbE}{{\mathbb E}} \newcommand{\bbN}{{\mathbb N}}
\newcommand{\bbM}{{\mathbb M}} \newcommand{\bbV}{{\mathbb V}}
\newcommand{\bbC}{{\mathbb K}} \newcommand{\bbF}{{\mathbb F}}
\newcommand{\bbP}{{\mathbb P}}


%\newcommand{\zigzag}[2]{\hspace*{#1cm}
%\begin{picture}(0,0)
%    \put(0,.1){\pszigzag[linewidth=.2mm,
%                         coilwidth=.2cm,
%                         coilheight=.9,
%                         coilarm=0cm](#2,0)}
%\end{picture}}
         

\newcommand{\dessin}[4]{
\begin{figure}[htb]
\centering
\includegraphics[scale= #2]{#1}
\caption{#3}
\label{#4}
\end{figure}}

\newcommand{\dessinsscaption}[2]{
\begin{figure}[htb]
\centering
\includegraphics[scale= #2]{#1}
\end{figure}}
\begin{document}

\title{ECSTASY OF AGONY}

Info-SUP~: $\text{A}_2$, Promo 2007\\
\vspace{1cm}
\begin{center}
\hrule \vspace{0.5cm}
  \textbf{\huge{ECSTASY OF AGONY}}\\
\vspace{0.5cm}\hrule
  \vspace{1cm}
  \textbf{Quatri\`eme Soutenance} \\
  \vspace{2cm}
  \dessinsscaption{pont}{0.41}
  \vspace{2cm}

\textsc{Quadrat Quentin}, quadra\_q\\
\textsc{Kadiri Anass}, kadiri\_a\\

\end{center}
\newpage
\maketitle

\tableofcontents
\newpage

\section{Introduction}

Ecstasy est un projet r\'ealis\'e en OpenGl et Delphi. Il comprend
deux parties : -- la premi\`ere partie \`a pour but de simuler le plus
pr\'ecisement possible la dynamique d'une voiture, -- la deuxi\`eme de
repr\'esenter en 3D la circulation automobile d'une ville de type
am\'ericaine (c'est \`a dire ayant une forme g\'eom\'etrique
r\'eguli\`ere).

Cette id\'ee a \'et\'e inspir\'ee par des jeux tels que GTA ou
Midtowm Madness et un projet scientifique appel\'e STARDUST \cite{STARDUST}
(Towards Sustainable Town Development : A Research on Deployment of
Urban Sustainable Transport)
%\\footnote{ Pour plus d'informations,
 %voir le site : {\tt http://www.soton.ac.uk/~trawww/stardust}.}.
dont l'objectif est de simuler une ville et ses v\'ehicules,
pour  en \'etudier les probl\`emes de circulation.


\section{Cahier des charges}

Voici, le cahier des charges avec tous ses objectifs que nous nous
\'etions fix\'es. On peut remarquer, que presque tous 
ont \'et\'e respect\'es (sachant qu'une personne a abandonn\'e ce projet en cours
de route).

\subsection{Objet de l'\'etude}
Ce projet a plusieurs objectifs.
%\begin{itemize}

%\item

Le premier objectif est d'apprendre \`a travailler en groupe,
et \`a transmettre nos id\'ees de fa\c{c}on
claire. Un groupe anarchique, o\`u il y aurait quatre
chefs au lieu d'un, est un groupe qui ne survit g\'en\'eralement pas
longtemps. Il faudra apprendre \`a concillier les attentes et les
d\'esirs de chacun et donc d'apprendre \`a g\'erer au mieux les
capacit\'es de chacun.

%\item

Le deuxi\`eme objectif est d'apprendre sous Delphi, le langage Pascal, la
manipulation de la 3D sous OpenGL et 3D Studio Max et le son avec Fmod. Aussi,
sous Emacs, nous utiliserons \LaTeX\ pour r\'ediger les cahiers de soutenance.
Enfin, il faudra cr\'eer notre prope page WEB gr\^ace \`a DreamWeaver.

\subsection{D\'ecoupage du projet}
Le projet se d\'ecoupe en quatre grandes parties :
\begin{itemize}
\item Construction d'un univers 3D (terrain, ville).
\item
R\'ealisation d'une dynamique des voitures la plus r\'ealiste
possible.
\item  R\'ealisation d'une logique des voitures ennemies
et neutres.
\item  Finalisation du jeu pour une meilleure
jouabilit\'e (effets sp\'eciaux, sons, menus).
\end{itemize}

Dans un premier temps, la cr\'eation d'une base de donn\'ees des
diff\'erentes voitures et immeubles,
pr\'ealablement dessin\'es sous 3DStudio Max, vont permettre la
cr\'eation d'un univers plus complexe au fur et \`a
mesure des soutenances.  Dans un premier
temps, gr\^ace \`a la dynamique du v\'ehicule, la voiture du joueur pourra
joyeuseument 'gambader' sur les terrains caillouteux pour tester
ses nouvelles suspensions et faire de joyeuses embard\'ees pour
s'encastrer dans le d\'ecor. Dans un second temps, de nombreuses voitures
neutres vont parcourir les  routes caboss\'ees sur lesquelles
des
voitures ennemies-amies vont se poursuivre par des champs
attractifs/r\'epulsifs. On tentera d'\'egayer le jeu par
des tirs, des effets sp\'eciaux gr\^ace \`a OpenGL et des effets sonores
gr\^ace \`a Fmod. On  y int\'egrera un menu cr\'e\'e par Delphi
permettant de lancer le jeu et de le configurer.

En r\'esum\'e, nous avons :

\begin{itemize}

\item  \emph{Construction d'un monde al\'eatoire :}

Cr\'eation d'un terrain al\'eatoire, construction d'une ville
avec des blocs d'immeubles et des routes. Le tout g\'en\'er\'e
par programmation.

El\`eve responsable du bon fonctionnement de cette partie : Kadiri Anass

\item  \emph{Dynamique des v\'ehicules :}

Mise en place de la dynamique des voitures, c'est-\`a-dire de
l'acc\'el\'e\-ration, des vitesses, des frottements, des suspensions des roues
par rapport \`a la hauteur du terrain et des rotations des roues (avant,
arri\`ere, gauche, droite). Collisions entre les v\'ehicules et le d\'ecor
ainsi que des voitures entre elles.

El\`eve responsable du bon fonctionnement de cette partie : Quadrat Quentin.

\item  \emph{Sc\'enarisation et logique des voitures :}

Mise en place d'un sc\'enario : comment les voitures ennemies et
amies vont se poursuivre (champs attractifs, r\'epulsifs, tirs,
d\'eplacement des voitures neutres).

El\`eve responsable du bon fonctionnement de cette partie : Ngombe Jessica.

\item  \emph{Finalisation du projet :}

Cr\'eation des menus (fen\^etres Delphi), mise en place des sons,
mod\'eli\-sations des voitures et des immeubles, effets sp\'eciaux avec
OpenGl (tirs, explosions ...).

El\`eve responsable du bon fonctionnement de cette partie : tout le monde.
\end{itemize}

\subsection{Premi\`ere soutenance}

Cr\'eation d'un exportateur de fichier ASCII de 3DStudio Max
utilisable pour OpenGl.  Chacun va se sp\'ecialiser dans un domaine :
chaque personne va concevoir des objets \'el\'ementaires (une personne
pour cr\'eer des batiments, une autre pour les blocs de voitures et
les blocs de paysages pour la derni\`ere).  Ceci va c\'eer notre base
de donn\'ee d'objets.  Dans un deuxi\`eme temps, tout le monde va
apprendre \`a utiliser 3DStudio Max en fonction du loader fraichement
cr\'eer, puis apprendre \`a programmer OpenGl et Pascal avec le
loader.

\vspace{0.5 cm}
\begin{tabular}{|c||c|c|c|}
\hline
Actions & Quadrat & Kadiri & N'gombe \\ \hline \hline
Loader ASE vers Opengl & X & - & -  \\ \hline
Apprendre la syntaxe Pascal & - & X & X \\ \hline
Premiers exemples en Delphi& - & X & X \\ \hline
Apprendre Opengl & X  & X & X  \\ \hline
Premiers exemples en OpenGl & X & X & X \\ \hline
HTML et \LaTeX\ & X & X & X \\  \hline
\end{tabular}
\vspace{0.5 cm}

\subsection{Deuxi\`eme soutenance}

Assemblage de la ville par programmation. Une personne s'occupe de
l'assemblage des bouts de batiments pour donner au final des blocs de
batiments. Une autre pour cr\'eer des voitures et une derniere pour
cr\'eer des paysages 3D.  La premi\`ere voiture articul\'ee va \^etre
cr\'e\'ee. Mais, il n'y aura pas encore de d\'eplacement.

\vspace{0.5 cm}
\begin{tabular}{|c||c|c|c|}
\hline
Actions & Quadrat & Kadiri & N'gombe \\ \hline \hline
Articulation des voitures & X & - & - \\  \hline
Cr\'eation des voitures & X & - & - \\  \hline
Cr\'eation d'une ville simple & - & X & - \\  \hline
Cr\'eation de routes simples & - & - & X \\  \hline
Cr\'eation de paysages & - & - & -  \\  \hline
HTML et \LaTeX\ & X & X & X  \\  \hline
\end{tabular}
\vspace{0.5 cm}

\subsection{Troisi\`eme soutenance}

Au moins une voiture articul\'ee se d\'eplace dans la ville.
Cr\'eation de la logique des voitures (strat\'egie, poursuites) sur un plan 2D.
Sons, menus, effets sp\'eciaux.

\vspace{0.5 cm}
\begin{tabular}{|c||c|c|c|}
\hline
Actions & Quadrat & Kadiri & N'gombe  \\ \hline \hline
D\'eplacement d'une voiture & X & - & - \\  \hline
Logique des voitures neutres & X & X & - \\  \hline
Logique des voitures ennemies  & - & - & X \\  \hline
Effets sp\'eciaux avec OpenGl & X & X & X \\ \hline
Cr\'eation de menus & - & - & X \\ \hline
Sauvegarde et chargements & - & - & X \\ \hline
Sons & - & X & - \\ \hline
HTML et \LaTeX\ & - & - & X \\  \hline
\end{tabular}
\vspace{0.5 cm}

\subsection{Quatri\`eme soutenance}

Strat\'egie finie, assemblage de tous les \'el\'ements. Tirs, explosions. Finiolage, am\'elioration de la
beaut\'e du jeu et relecture des bogues. Cr\'eation du CD, site web, version light, install Shield.

\vspace{0.5 cm}
\begin{tabular}{|c||c|c|c|}
\hline
Actions & Quadrat & Kadiri & N'gombe \\ \hline \hline
Introduction AVI & - & - & X \\ \hline
D\'eplacements des voitures & X & X & - \\  \hline
Tirs & X & X & -  \\  \hline
Pauffinage du jeu & X & X & X \\  \hline
D\'eboggage & X & X & X \\  \hline
Am\'eliorations des dessins & X & X & X \\  \hline
Install Schield  & - & - & X \\  \hline
CD, pochette, web, papiers  & - & - & X \\  \hline
\end{tabular}
\vspace{0.5 cm}

\begin{tabular}{|c|c|}
\hline Actions \\ \hline \hline
Loader ASE vers Opengl \\ \hline
D\'eplacement du joueur \\  \hline
Gestions des collisions \\  \hline
Cr\'eation du terrain 3D \\ \hline
Cr\'eation des blocs d'immeubles dans l'espace \\
\hline
Cr\'eation des routes dans l'espace \\ \hline
Animation des ennmis \\  \hline
Animation des neutres \\  \hline
Animation des armes \\\hline
Menus, parametrages, introduction \\  \hline
Sauvegarde, chargement du jeu \\\hline
Cr\'eation du terrain \\  \hline
Touches et sons \\ \hline
Dessins et bibliot\'eque d'objets
Effets sp\'eciaux avec OpenGl \\
\hline
\end{tabular}
\vspace{0.5 cm}

\subsection{Conclusion du projet}

Ce projet va permettre pour la plupart d'entre nous de d\'ebuter en
informatique et de concevoir un jeu vid\'eo. Ceci va nous contraindre \`a des
d\'emarches personelles pour apprendre un nouveau langage de programation
qu'est le Pascal, et la mod\'eli\-sation 3D gr\^{a}ce \`a OpenGL et 3D Studio Max. Ce
projet nous permettra donc d'acq\'erir de l'\'exp\'erience au niveau de la
programmation et d'appliquer efficacement nos cours re\c{c}us en classe.

\section{Ce qui a \'et\'e r\'ealis\'e dans ce projet}
\subsection{La voiture}
Lors de la premi\`ere soutenance, notre voiture (mod\'elis\'ee en deux
dimensions) \'etait constitu\'ee d'une masse ponctuelle \`a laquelle
\'etait accroch\'ee une roue par un ressort (cf section \ref{dyna2}).

Pour la deuxi\`eme soutenance, elle \'etait mod\'elis\'ee par une
barre de masse ponctuelle, plac\'ee en son centre, aux extr\'emit\'es
de laquelle \'etaient accroch\'ees deux roues par des ressorts. Les
roues poss\'edaient des pneus ayant un comportement \'elastique. Seul
le mouvement vertical de la voiture \'etait visualis\'e (cf section
\ref{dyna3}).

Pour la troisi\`eme soutenance la voiture command\'ee par le joueur
pouvait se d\'eplacer dans la ville en 3D . Le fonctionnement de la
suspension est visible lors des changements de pentes des routes.
Seul le tangage (cf section \ref{dyna4}) avait un sens m\'ecanique.

Pour la quatri\`eme soutenance  la dynamique du mouvement horizontal 
a \'et\'e r\'ealis\'ee.  Pour faire avancer la voiture, le
joueur donne des consignes de direction et de puissance du moteur,
une dynamique du mouvement horizontal d'une voiture est simul\'ee
de fa\c{c}on tr\`es simplifi\'ee (cf section \ref{dyna4}). En particulier
l'inertie de rotation est repr\'esent\'e, par contre les frottements avec
le sol sont tr\`es simplifi\'es.

\subsection{La structure de la ville}
L'\'elaboration de la ville n'a commenc\'ee que pour la deuxi\`eme
soutenance. La structure de la ville constitu\'ee de blocs,
eux-m\^emes constitu\'es de batiments (tir\'es au hasard) et de routes
a \'et\'e d\'efinie (cf section \ref{ville}).  Les routes ne
poss\'edaient pas de textures et les immeubles \'etaient repr\'esent\'es
par des cubes de tailles enti\`eres entre 1 et $n$.

Le travail pour la troisi\`eme soutenance a permis d'achever
l'infrastructure de la ville : -- textures au format bmp, tga ou jpeg
des diff\'erentes routes, -- am\'elioration de la structure des
immeubles (dessins, positionnements, tailles), -- ajout du fleuve et
des ponts, -- ajout de la signalisation des routes, -- ajout d'effets
de lumi\`eres (brume, \'eclairage) (cf section
\,\ref{fog},\,\ref{blend},\, \ref{Lumieres}). Les immeubles sont tir\'es au
hasard dans un sous ensemble de la biblioth\'eque (choisie en fonction
de la place disponible) jusqu'\`a remplir totalement la rang\'ee (ce
qui est possible \`a cause de la pr\'esence obligatoire d'un immeuble
de taille un).

Le travail pour la derni\`ere soutenance, a \'et\'e d'am\'eliorer les
bords de la ville.  On l'a rendu sans bord en la mettant sur un tore
2D. La ville peut donc \^etre vue comme de taille infini mais
p\'eriodique (cf section \ref{ville}).  L'aspect des immeubles a
\'et\'e am\'elior\'e en rajoutant des textures pour le jour et pour la
nuit. Les particules des feux tricolores ont \'et\'e arrang\'ees, car
lors de la deuxi\`eme soutenance car elles ne s'affichaient pas
correctement. Les plans de feux ont \'et\'e mis
en place.


\subsection{La circulation}
La circulation n'a \'et\'e impl\'ement\'ee que pour la derni\`ere
soutenance . La ville est d\'ecoup\'ee en blocs r\'eguliers.  Chaque
bloc poss\`ede 2 routes \`a 4 voies. A chaque voie est associ\'e la
file de voiture qu'elle contient . Les voitures se d\'eplacent et
s'arr\^etent aux feux rouges (cf section section \ref{circulation}).
Chaque voiture est simul\'ee par un syst\`eme de onze \'equations
diff\'erentielles.  Le jeu a fonctionn\'e avec 432 voitures dans une
viile de 36 blocs \`a 18 images 1600x1200 par seconde sur un PC Athlon
1.8Ghz pss\'edant un carte graphique Radeon 8500.

\subsection{Utilisation de OpenGL}
Un chargeur pour OpenGL d'objets 3D --- cr\'e\'e dans 3D Studio Max
(ASE) --- a \'et\'e r\'ealis\'e pour la premi\`ere soutenance. Cela a
permis la cr\'eation de carcasses de voitures sophistiqu\'ees (cf
section \ref{ASE}).

Pour la deuxi\`eme soutenance une d\'ebauche de cr\'eation
de terrains 3D en fil de fer (plus couramment appel\'e height map) a \'et\'e faite.
 Cette structure n'a pas \'et\'e utilis\'ee dans la suite
par manque de temps (cf section de \ref{terrain}).

Pour la troisi\`eme soutenance, on a \'etudi\'e comment mettre en
place les lumi\`eres, la brume et la transparence dans notre ville (cf
section de \ref{Lumieres} \ref{blend}, \ref{fog}). 
Ensuite, on a impl\'ement\'e un petit syst\`eme de particules
pour simuler les eclaboussures.

\subsection{Utilitaires}
Pour la deuxi\`eme soutenance, on a cr\'e\'e un menu en Delphi pour
prendre en compte des diff\'erents param\`etres du jeu. Pour la
troisi\`eme soutenance, on a mis en place la vid\'eo d'introduction et
le son avec FMod puis DirectSound.

Pour la derni\`ere soutenance, on a d\'ebogu\'e, nettoyer et commenter
nos codes sources . On a fait ainsi de gros gains de temps de
calcul. On a \'egalement \'etudi\'e la notion
d'h\'eritage des objets et on a am\'elir\'e les autorisations d'acc\'es aux champs  des objets.


\section{La ville}
\subsection{Introduction}
Notre ville est de type am\'ericaine, c'est \`a dire qu'elle est
constitu\'ee de blocs r\'eguliers s\'epar\'es par des routes nord-sud
et est-ouest. Pour \'eviter la description du comportement du bord de
la ville, elle a \'et\'e plac\'ee sur un tore 2D. Ce qui signifie
qu'elle est infinie et p\'eriodique. Dans cette ville, circulent des
voitures dans les deux sens sur les art\`eres nord-sud et est-ouest,
munies de feux de circulation. Une voiture guid\'ee par le joueur se
d\'eplace dans la ville. Le comportement m\'ecanique de chaque voiture
est d\'ecrite par un syst\`eme de onze \'equations diff\'erentielles.


\dessin{structure}{0.4}{Le squelette du projet}{structure}


\subsection{Le bloc}\label{ville}
La ville est un tableau $n \times n$ de blocs.  Chaque bloc de la
ville, figure \ref{bloc}, est une classe Delphi compos\'ee de :
\begin{itemize}
\item un bloc de maisons tir\'ees au hasard dans une biblioth\`eque
  --- r\'ealis\'ee dans 3DSMax et import\'ee gr\^ace au chargeur ASE.
  Il est compos\'e de quatre alignements de maisons de
  tailles diff\'erentes autour d'une cour int\'erieure. 
\item deux routes de quatre voies qui ne sont pas horizontales;
\item un carrefour horizontal avec une altitude et sa signalisation
  constitu\'e de quatre feux;
\item deux tableaux de quatre files de voitures contenant des pointeurs
  vers les voitures circulant sur les routes du bloc.
\end{itemize}

Il est affich\'e au moyen d'une liste d'affichage OpenGL.

\dessin{ville}{0.9}{Mod\`ele d'un bloc  de la ville}{bloc}
 
Une maison est une classe  qui contient sa taille, sa position et un
num\'ero d'indentification d'une maison import\'ee de 3D studio Max. 

Une route est un enregistrement qui contient les coordonn\'ees de ses
quatre sommets. 

Le type carrefour est constitu\'e d'un type route et de quatre feux
tricolores.  Une fonction r\'ealise le plan de feux du blocs et
affichent des textures color\'ees (rouge, jaune ou verte). Ces
textures sont particuli\`eres puisqu'elles utilisent la transparence
d'OpenGL (conf\'ere section \ref{blend}).

Chaque listes chain\'ees de type TFile est associ\'e \`a une voie de
circulation. Chaque rue \`a quatre voies est \`a deux sens. Dans
l\'etat actuel du jeu, les voitures autres que le joueur restent sur
leur voie mais s'arr\^etent aux feux (cf section \ref{circulation}).

Cette structure r\'eguli\`ere est tr\'es utile pour l'optimisation de
la vitesse du jeu. 
On peut savoir facilement la case dans laquelle se trouve une voiture.
Il suffit de faire la division enti\`ere de sa position
par la taille du bloc pour obtenir le num\'ero de la case.
  
%Pour tester les collisions du joueur avec les immeubles, il suffit
%simplement de tester l'appartenance de la voiture au rectangle, form\'e
%par les quatres rang\'ees de batiments, du bloc dans lequel se trouve la
%voiture, plut\^ot que de tester l'appartenance \`a un immeuble
%quelconque de la ville enti\`ere.


Le fait que les voitures soient attach\'ees au bloc permet des gains
de temps d'affichage. En effet, dans le pire des cas, seul neuf blocs
sont affich\'es : le bloc du joueur et les huit connexes en
utilisant un algorithme permettant d'\'eviter de calculer les objets
n'appartenant pas au c\^one de vision (cf section \ref{frustum}).



% Dans la ville, la voiture \'etait repr\'esent\'ee par un rep\`ere, ne
% pouvait tourner sur elle-m\^eme et restait dans le plan horizontal.
% Actuellement, elle est modelis\'ee en 3D par une carcasse et quatre roues
% ind\'ependantes et une dynamique repr\'esentant le tangage de la
% voiture par un syt\`eme de trois \'equations diff\'erentielles. Le
% mouvement horizontal de la voiture et command\'ee par le clavier
% (vitesse et direction). La simulation d'une chute \'eventuelle de la
% voiture dans le canal a \'et\'e impl\'eement\'e.


\section{La dynamique des voitures}\label{Voitures}
La partie la plus difficile th\'eoriquement est la mod\'elisation
de la m\'ecanique de la voiture incluant une suspension
simplifi\'ee. Le tangage et le mouvement horizontal ont \'et\'e
mod\'elis\'es. Le roulis a \'et\'e n\'eglig\'e.

\dessin{suspension}{0.4}{Une Audi Monster}{monster}

% Lors de la premi\`ere et deuxi\`eme soutenace, la dynamique \'etait
% visualis\'ee par des courbes.

% Lors de la deuxi\`eme soutenace, Dans la ville 3D, la voiture \'etait
% mod\'elis\'ee par un rep\`ere, ne pouvait tourner sur elle-m\^eme et
% restait dans le plan horizontal.  

% Depuis la soutenance trois, elle est modelis\'ee en 3D par une
% carcasse et quatre roues ind\'ependantes et une dynamique
% repr\'esentant le tangage de la voiture par un syt\`eme de trois
% \'equations diff\'erentielles du second ordre. Le mouvement horizontal
% de la voiture du joueur et command\'ee par le clavier (vitesse et
% direction). La simulation d'une chute \'eventuelle de la voiture dans
% le canal a \'et\'e impl\'ement\'ee. La voiture pouvait tourner autour
% de son centre de gravit\'e de fa\c{c}on arbitraire.

% Maintenant, pour faire avancer sa voiture, le joueur doit tourner le
% volant et donner la puissance du moteur (avec la souris ou au clavier)
% ce qui modifie l'angle les deux roues de devant et donne
% l'acc\'el\'eration de la voiture. La carcasse d\'ecrit un mouvemen
% tenant compte de sa masse et son moment inertie autour de la verticale
% et en tangage.  Elle est devenue difficile \`a piloter et toutes les
% voitures n'ont pas le m\^eme comportement.

% Enfin, bien que les voitures autres que celle du joueur ont des
% trajectoires simples, dans l'\'tat actuel du jeu, elles sont
% mod\'elis\'ees m\'ecaniquement de le m\^eme fa\c{c}on.

\subsection{Principe de la moindre action}\label{dyna1}

\subsubsection{D\'efinition}
On appelle action $\AAA$ d'un syst\`eme m\'ecanique l'int\'egrale
le long du mouvement de la diff\'erence de son \'energie cin\'etique et
de son \'energie potentielle:
$$\AAA(x()) = \int (\EE_c(x(t)) - \EE_p(x(t))dt\;,$$
o\`u  $\EE_c(x(t))$ designe l'\'energie cin\'etique, $\EE_p(x(t))$
l'\'energie potentielle et $t\mapsto x(t)$ la trajectoire du syst\`eme.

Le principe de la moindre action nous dit que la trajectoire
du syst\`eme est celle qui minimise l'action.

Pour trouver cette trajectoire on calcule la variation de
l'action $\delta \AAA$ associ\'ee \`a la variation de la trajectoire
$\delta x$ et on d\'etermine les conditions qui assurent que
$\delta \AAA$ soit nul quelque soit $\delta x$.

\subsubsection{Exemple d'application}
Prenons le cas de deux masses, de poids respectifs $m_1$ et $m_2$,
accroch\'ees l'une \`a l'autre par un ressort de force $F =
-kl$ ou $l$ d\'esigne l'allongement du ressort ($|x_1(t)-x_2(t)|$)
avec  $x_1(t)$ et $x_2(t)$ les positions \`a l'instant $t$ des deux masses.

\begin{align*}
\delta \AAA =& 1/2\int (m_1(\dot{x_1}+\delta \dot{x}_1)^2+
 m_2( \dot{x}_2+\delta \dot{x}_2)^2-
 k(x_2+\delta x_2-x_1-\delta x_1)^2) dt\\
 &-1/2\int (m_1\dot{x}_1^2+m_2\dot{x}_2^2-k(x_2-x_1)^2)dt \;,
\end{align*}
$$\delta \AAA = \int m_1\dot{x}_1\delta \dot{x}_1+m_2\dot{x}_2 \delta
\dot{x}_2-k(x_2-x_1) (\delta x_2-\delta x_1)+o(\| x_1-x_2\|)\;,$$

Par int\'egration par partie on obtient :
$$\delta \AAA = \int -m_1\ddot{x_1}\delta x_1 -m_2\ddot{x_2}\delta x_2
- k(x_2-x_1)(\delta x_2-\delta x_1) + o(\| x_1-x_2\|)\;,$$
car on
suppose que les variations des trajectoires sont nulles aux
extr\'emites.

Finalement on trouve :
$$\delta \AAA =\int (-m_1\ddot{x_1}+k(x_2-x_1))\delta x_1 + \int
(-m_2\ddot{x_2}+k(x_1-x_2))\delta x_2 + o(\| x_1-x_2\|)\;.$$

L'action doit \^etre minimale c.a.d. $$\delta A = 0,\quad\forall
\delta x_1,\;\forall \delta x_2,$$

donc :
$$-m_1\ddot{x_1}+k(x_2-x_1) = 0,\quad-m_2\ddot{x_2}+k(x_1-x_2) = 0\;.$$

On obtient les m\^emes \'equations que celles en appliquant la loi
fondamentale de la dynamique $m\vec{\gamma} = \vec{F}$.

\subsection{Mouvement vertical d'un monocycle}\label{dyna2}

\subsubsection{Pr\'esentation du mod\'ele}
Le v\'ehicule est mod\'elis\'e en 2D, par une carcasse de masse
ponctuelle $M_v$ accroch\'ee \`a une roue (de rayon $r$ et de masse
$M_r$) par un ressort. On note $u(t)$ la l'altitude du sol par rapport
\`a au rep\'ere, $y(t)$ est l'altitude de la carcasse, $z(t)$
l'alongement du ressort, et $y(t)+z(t)$, l'altitude de la roue. On
note $g$ la gravit\'e (figure \ref{dyna}).

\dessin{dynavoit}{0.6}{Le monocycle}{dyna}

\subsubsection{Calcul des \'equations du mouvement}
Les forces qui sont en jeux sont : la pesanteur des masses (roue et
carcasse), la r\'epulsion du sol sur la roue et la force du ressort.

L'\'energie cin\'etique de la roue (not\'ee $\EE_r^c$) est :
$\EE_r^c = 1/2M_r(\dot{y}+\dot{z})^2\;.$

L'\'energie cin\'etique de la voiture (not\'ee $\EE_v^c$) est :
$\EE_v^c = 1/2M_v\dot{y}^2\;.$

L'\'energie ressort (not\'ee $\EE_r$) est :
$\EE_r = 1/2kz^2;.$

L'\'energie potentiel de la voiture (not\'ee $\EE_v^p$) est :
$\EE_v^p = M_vgy\;.$

L'\'energie potentiel de la roue (not\'ee $\EE_r^p$) est :
$\EE_r^p = M_rg(y+z)\;.$

L'\'energie de reaction du sol (not\'ee $\EE_s$) est : $\EE_s =
1/2((u-(y+z-r))^+)^2\;,$ c'est \`a dire que $\EE_s$ vaut $1/2(u-(y+z-r))^2$
quand $u-(y+z-r) > 0$, sinon il vaut $0$.

\begin{align*}
\AAA(x()) =& \int (\;1/2M_r(\dot{y}+\dot{z})^2 + 1/2M_v\dot{y}^2 -1/2kz^2 -M_vgy - M_rg(y+z)\\
&- 1/2(u-(y+z-r))^+)^2\;)dt\;
\end{align*}

Comme dans la sous-section pr\'ec\'edente expliquant le principe de la
moindre action, on trouve un syst\`eme d'\'equation diff\'erentielle
o\`u les inconnues sont l'altitude de la roue et de la carcasse :
\begin{align}
  &(\ddot{z}+\ddot{y})M_r+\ddot{y}M_v + g(M_v+M_r) - (u-(y+z-r))^+ = 0\;,\label{e1}\\
  &(\ddot{z}+\ddot{y})M_r+M_rg+kz-(u-(y+z-r))^+ = 0\;.\label{e2}
\end{align}
En notant : $w=y+z$, (\ref{e1})-(\ref{e2}) donne :
$$\ddot{y}M_v+gM_v-k(w-y)=0.$$
(\ref{e2}) donne :
$$\ddot{w}M_r+gM_r+k(w-y)-(u-(w-r))^+ = 0$$

\subsubsection{Discr\'etisation des \'equations diff\'erentielles}
Pour calculer les trajectoires des deux corps, nous pouvons approximer
les \'equations diff\'erentielles (\ref{e1}) et (\ref{e2}) par les
\'equations r\'ecurrentes, o\`u $h$ d\'esigne le pas de discr\'etisation~:

\begin{align*}
& y(t+h)=2y(t)-y(t-h)+gh^2-\frac{k(w-y)h^2}{M_v}\;,\\
& w(t+h)=2w(t)-w(t-h)+\frac{h^2}{M_r}(u(t)-w(t)+r)^+-\frac{h^2k(w(t)-y(t))}{M_r}-h^2g\;.
\end{align*}

\subsection{Mouvement vertical d'une moto (tangage)}\label{dyna3}

\subsubsection{Pr\'esentation du mod\'ele}
Le v\'ehicule est mod\'elis\'e en 2D, par une carcasse repr\'esent\'e
par une barre de demi longueur $l$ et de masse ponctuelle $M$ \`a
laquelle sont accroch\'ees deux roues (de rayon $R$ et de masse $m$)
par des ressorts. On note $u(t)$ la l'altitude du sol, $y(t)$ est
l'altitude de la carcasse, $y_1(t)$ et $y_2(t)$ les allongements des
deux ressorts, $\theta$ le degr\'e d'inclinaison du v\'ehicule.  On
note $g$ la gravit\'e, $\theta$ le degr\'e de penchement du v\'ehicule
(figure \ref{dyna}).

\dessin{dyna}{0.5}{Mod\'elisation de la voiture}{dyna}

\subsubsection{Calcul des \'equations du mouvement}
Les forces qui sont en jeux sont : la pesanteur des masses (roue et
carcasse), la r\'epulsion du sol sur les roues et la force des ressorts.

L'\'energie cin\'etique de la voiture est : $$\frac{M\dot{y}^2}{2}\;.$$

L'\'energie potentielle de la voiture  est : $$Mgy\;.$$

L'\'energie cin\'etique verticale de la roue de devant est
$(\dot{y_2}+l\cos\theta\dot{\theta}+\dot{y})^2\;, $
%On approximera ($\theta$ petit) l'\'energie cin\'etique verticale de la roue de devant par:
que l'on approxime en faisant l'hypoth\'ese $\theta$ petit par :
$$1/2m(\dot{y}_1+\dot{y}+l\dot{\theta})^2\;.$$

De m\^eme, l'\'energie cin\'etique verticale de la roue de derri\`ere est :
$$1/2m(\dot{y}_2+\dot{y}-l\dot{\theta})^2\;.$$


L'\'energie potentielle due \`a la pesanteur des deux roues est :
$$mg(2y+y_2+y_1)\;.$$

L'\'energie potentielle du ressort de la roue avant est :
$$1/2ky_1^2\;.$$


L'\'energie potentielle du ressort de la roue arri\`ere est :
$$1/2ky_2^2\;.$$

L'\'energie potentielle de r\'eaction du sol sur la roue de devant est
:
$$1/2([u(x+l)-(y_1+y+l\theta-R)]^+)^2\;,$$
o\`u $A^+$ d\'esigne la partie positive de $A$.

L'\'energie potentielle de r\'eaction du sol sur la roue de derri\`ere est :
$$1/2([u(x-l)-(y_2+y-l\theta-R)]^+)^2\;.$$


L'action \`a minimiser vaut donc :
\begin{align*}
\AAA=1/2\int &\big\{ M\dot{y}^2+m(\dot{y}_1+\dot{y}+l\dot{\theta})^2+
m(\dot{y}_2+\dot{y}-l\dot{\theta})^2 \\
&-ky_1^2-ky_2^2-2Mgy-2mg(2y+y_2+y_1) \\
&-([u(x+l)-(y_1+y+l\theta-R)]^+)^2\\
&-([u(x-l)-(y_2+y-l\theta-R)]^+)^2 \big\} dt\;
\end{align*}

On trouve un syst\`eme
d'\'equation diff\'erentielle o\`u les inconnues sont : trois
altitudes (une pour la carcasse, une pour chaque roue)
et enfin l'inclinaison de la carcasse ($\theta$).

On note :
$$R_1 =[u(x+l)-(y_1+y+l\theta-R)]^+\;,$$
$$R_2 =[u(x-l)-(y_2+y-l\theta-R)]^+\;,$$

On a :
\begin{align}
(M+2m)\ddot{y}+m\ddot{y}_1+m\ddot{y}_2 & = -2g(2m+M)+R_1+R_2\label{eqy}\;,\\
m(\ddot{y}_1+\ddot{y}+l\ddot{\theta}) &  = -ky_1-gm+R_1\label{eq1}\;,\\
m(\ddot{y}_2+\ddot{y}-l\ddot{\theta}) & = -ky_2-gm+R_2\label{eq2}\;,\\
m(\ddot{y}_1-\ddot{y}_2+2l\ddot{\theta}) & = R_1-R_2\label{eqo}\;.
\end{align}

En faisant  (\ref{eqo}) plus (\ref{eq2}) moins (\ref{eq1}) on obtient
$0 =k(y_1-y_2)$
et donc $y_1=y_2.$

En faisant  (\ref{eqy}) moins (\ref{eq1}) moins (\ref{eq2}) on obtient :
\begin{equation}
M\ddot{y}=-gM+2ky_1\;.\label{eqyy}
\end{equation}

L'equation (\ref{eqo}) donne alors :
\begin{equation}
2ml\ddot{\theta}= R_1-R_2\;. \label{eqoo}
\end{equation}

Puis, (\ref{eq1}) moins $\frac{m}{M}$(\ref{eqyy}) moins $\frac{1}{2}$(\ref{eqoo})
donne :
\begin{equation}
m\ddot{y}_1 = -ky_1(1+\frac{2m}{M})+\frac{R_1+R_2}{2}\;.
\end{equation}

%Enfin, (\ref{eq2}) moins $\frac{m}{M}$(\ref{eqyy}) moins $\frac{1}{4}$(\ref{eqoo})
%donne :
%\begin{equation}
%m\ddot{y}_2 = \frac{k(y_1-y_2)}{2}+R_2+\frac{mg(2m+M)}{M}\;.
%\end{equation}

Finalement, on obtient, le syst\`eme alg\'ebriquo-dif\'erentiel suivant :
\begin{align*}
%M\ddot{y} &=-gM+2ky_1\;,\\
%m\ddot{y}_1 &= -ky_1(1+\frac{2m}{M})+\frac{R_1+R_2}{2}\;,\\
%m\ddot{y}_2 &= -ky_2(1+\frac{2m}{M})+\frac{R_1+R_2}{2}\;,\\
%2ml\ddot{\theta} &= (R_1+R_2)\;.
\ddot{y} &=-2g+\frac{2ky_1}{M}\;,\\
\ddot{y}_1 &= -ky_1(\frac{1}{m}+\frac{2}{M})+\frac{R_1+R_2}{2m}\;,\\
y_2 &= y_1\;,\\
\ddot{\theta} &= \frac{R_1+R_2}{2ml}\;.
\end{align*}

%En notant :
%$$P=\begin{bmatrix}
%M+2m & m & m & 0 \\
%m & m & 0 & ml \\
%m & 0 & m & -ml \\
%0 & m & -m & 2ml \\
%\end{bmatrix}\;,$$
%le syst\`eme differentiel s'\'ecrit :
%$$\begin{bmatrix}
%\ddot{y} \\ \ddot{y_1} \\ \ddot{y_2} \\ \ddot{\theta}
%\end{bmatrix}
%= P^{-1}
%\begin{bmatrix}
%-g(2m+M) + R_1+R_2 \\
%-gm -ky_1+R_1\\

%-gm -ky_2+R_2\\
%R_1+R_2\\
%\end{bmatrix}\;.
%$$
%\subsection{Discr\'etisation des \'equations differentielles}

\subsubsection{Discr\'etisation des \'equations diff\'erentielles}
Pour calculer les trajectoires des corps, nous pouvons approximer
les \'equations diff\'erentielles par des
\'equations r\'ecurrentes, o\`u $h$ d\'esigne le pas de discr\'etisation en temps~:

\begin{align*}
y(t+h) &=2y(t)-y(t-h)+h^2\left(-2g+\frac{2ky_1}{M}\right)\;,\\
y_1(t+h) &=2y_1(t)-y_1(t-h)+h^2\left(-ky_1(\frac{1}{m}+\frac{2}{M})+\frac{R_1+R_2}{2m}\right)\;,\\
\theta(t+h) &=2\theta(t)-\theta(t-h)+h^2\left(\frac{R_1+R_2}{2ml}\right)\;.
\end{align*}

\subsection{Mouvement vertical d'une voiture}
\subsubsection{Pr\'esentation du mod\'ele}
On veut \'etudier le mouvement vertical d'une voiture dont le 
mouvement horizontal est donn\'e par le triplet $(x(t),y(t),\phi 
(t))$, o\`u $\phi (t)$ est la direction du v\'ehicule. Celui-ci est
mod\'elis\'e en 3D, par une carcasse repr\'esent\'ee
par une plaque (de longueur $2L$, de largeur $2l$, de masse
ponctuelle $M$ et de moment d'inertie $I_{\theta}$ et $I_{\alpha}$)
\`a laquelle sont accroch\'ees quatre roues (de rayon
$r$ et de masse $m$) par des ressorts (de rigidit\'e $k$).

On note  $I_{\alpha}$ le moment
d'inertie par rapport \`a l'axe longitudinal de la voiture (roulis),
$I_{\theta}$ le moment d'inertie
par rapport \`a l'axe transversal de la voiture (tangage), $g$ la gravit\'e,
$u(x,y)$ l'altitude du sol, $z(t)$ l'altitude du centre de gravit\'e
de la carcasse, $z_1(t) \ldots z_4(t)$ les
allongements des quatre ressorts de la suspension de la voiture, $\theta$
l'angle de tangage et $\alpha$ l'angle de roulis (conf\`ere \ref{imagevoit}).

\subsubsection{Calcul des \'energies cin\'etiques}
De la premi\`ere roue :
$ 1/2m(\dot{z}+L\dot{\theta}+\dot{z}_1+l\dot{\alpha})^2 \;.$

De la deuxi\`eme roue :
$ 1/2m(\dot{z}+L\dot{\theta}+\dot{z}_2-l\dot{\alpha})^2 \;.$

De la troisi\`eme roue :
$ 1/2m(\dot{z}-L\dot{\theta}+\dot{z}_3-l\dot{\alpha})^2 \;.$

De la quatri\`eme roue :
$ 1/2m(\dot{z}-L\dot{\theta}+\dot{z}_4+l\dot{\alpha})^2 \;.$

De la caracasse :
$1/2(M\dot{z}^2 +I_{\theta}\dot{\theta}^2 +I_{\alpha}\dot{\alpha}^2)\;.$


\subsubsection{Calcul des \'energies potentielles}
Des quatre roues :
$ mg(4z+z_1+z_2+z_3+z_4)\;.$

Des quatre ressorts :
$ 1/2k(z^2_1+z^2_2+z^2_3+z^2_4)\;.$

De la caracasse :
$ Mgz\;.$

Soit :
\begin{align*} 
R_1 &= (u(x+L\cos\phi+l\sin\phi,y+L\sin\phi-l\cos\phi)-(z+z_1+L\theta+l\alpha-r))^+\;;\\
R_2 &= (u(x+L\cos\phi-l\sin\phi,y+L\sin\phi+l\cos\phi)-(z+z_2+L\theta-l\alpha-r))^+\;;\\
R_3 &= (u(x-L\cos\phi-l\sin\phi,y-L\sin\phi+l\cos\phi)-(z+z_3-L\theta-l\alpha-r))^+\;;\\
R_4 &= (u(x-L\cos\phi+l\sin\phi,y-L\sin\phi-l\cos\phi)-(z+z_4-L\theta+l\alpha-r))^+\;.
\end{align*}

L'\'en\'ergie de la r\'eaction du sol sur la premi\`eme roue est alors
de $1/2R_1^2\;,$ celle de la deuxi\`eme roue : $1/2R_2^2\;,$ celle de
la troisi\`eme : $1/2R_3^2\,$ et enfin la quatri\`eme : $1/2R_4^2\;.$

\dessin{Voiture}{0.65}{Mod\'elisation de la voiture}{imagevoit}

\subsubsection{Calcul des \'equations des trajectoires}

On trouve :
\begin{align*}
\delta\AAA = \int &
                      M\dot{z}\delta \dot{z}
                      + I_{\theta}\dot{\theta}\delta\dot{\theta} +
                        I_{\alpha}\dot{\alpha}\delta\dot{\alpha} -2Mg\delta z \\
                  &   -2mg(4\delta z + \delta z_{1} + \delta z_{2}
                          +\delta z_{3} + \delta z_{4}) \\
                  &   -k(z_{1}\delta z_{1} + z_{2}\delta z_{2}
                      +z_{3}\delta z_{3} + z_{4}\delta z_{4}) \\
                  &   +m(\dot{z}+\dot{z}_1+L\dot{\theta}+l\dot{\alpha})
                      (\delta\dot{z}+\delta\dot{z}_1+L\delta\dot{\theta}+l\delta\dot{\alpha})
                      \\
                  &   +m(\dot{z}+\dot{z}_2+L\dot{\theta}-l\dot{\alpha})
                      (\delta\dot{z}+\delta\dot{z}_2+L\delta\dot{\theta}-l\delta\dot{\alpha})
                      \\
                  &   +m(\dot{z}+\dot{z}_3-L\dot{\theta}-l\dot{\alpha})
                      (\delta\dot{z}+\delta\dot{z}_3-L\delta\dot{\theta}-l\delta\dot{\alpha})
                      \\
                  &   +m(\dot{z}+\dot{z}_4-L\dot{\theta}+l\dot{\alpha})
                      (\delta\dot{z}+\delta\dot{z}_4-L\delta\dot{\theta}+l\delta\dot{\alpha})\\
                  &   +R_1(\delta z+\delta z_1+L\delta\theta+l\delta\alpha-r)\\
                  &   +R_2(\delta y+\delta y_2+L\delta\theta-l\delta\alpha-r)\\
                  &   +R_3(\delta y+\delta y_3-L\delta\theta-l\delta\alpha-r)\\
                  &   +R_4(\delta y+\delta y_4-L\delta\theta+l\delta\alpha-r)
\end{align*}

La variation de l'action apr\'es int\'egration par partie vaut :
\begin{align*}
\delta\AAA = \int &
                      -M\ddot{z}\delta z
                      - I_{\theta}\ddot{\theta}\delta\theta -
                        I_{\alpha}\ddot{\alpha}\delta\alpha -2Mg\delta z \\
                  &   -2mg(4\delta z + \delta z_{1} + \delta z_{2}
                          +\delta z_{3} + \delta z_{4}) \\
                  &   -kz_{1}\delta z_{1} -kz_{2}\delta z_{2}
                      -kz_{3}\delta z_{3} -kz_{4}\delta z_{4} \\
%avant de faire l'IPP
%                     &  +4m\dot{z}\delta\dot{z} + m\dot{z}(\delta\dot{z}_1
%                                                + \delta\dot{z}_2
%                                                + \delta\dot{z}_3
%                                                + \delta\dot{z}_4)
%                                                \\
%                    &   + m(\dot{z}_1\delta\dot{z}_1
%                        + \dot{z}_2\delta\dot{z}_2
%                        + \dot{z}_3\delta\dot{z}_3
%                        + \dot{z}_4\delta\dot{z}_4)\\
%                    &   + (\dot{z}_1 + \dot{z}_2 + \dot{z}_3 + \dot{z}_4
%                          ) m\delta\dot{z}\\
%                    &   + (\dot{z}_1+\dot{z}_2-\dot{z}_3-\dot{z}_4)
%                          mL\delta\dot{\theta} \\
%                    &   + (\dot{z}_1-\dot{z}_2-\dot{z}_3+\dot{z}_4)
%                          ml\delta\dot{\alpha} \\
%                    &   + (\delta\dot{z}_1+\delta\dot{z}_2
%                          -\delta\dot{z}_3-\delta\dot{z}_4)
%                                                mL\dot{\theta}\\
%                    &   +4mL^2\dot{\theta}\delta\dot{\theta}
%                        +4ml^2\dot{\alpha}\delta\dot{\alpha}\\
%                    &   +(\delta\dot{z}_1-\delta\dot{z}_2
%                        -\delta\dot{z}_3+\delta\dot{z}_4)
%                        ml\dot{\alpha}\\
%END
                    &   -4m\ddot{z}\delta z - m\ddot{z}(\delta z_1
                                               +\delta z_2
                                               +\delta z_3
                                               +\delta z_4)
                                               \\
                   &   - m\ddot{z}_1\delta z_1
                       - m\ddot{z}_2\delta z_2
                       - m\ddot{z}_3\delta z_3
                       - m\ddot{z}_4\delta z_4\\
                   &   - (\ddot{z}_1 + \ddot{z}_2 + \ddot{z}_3 + \ddot{z}_4
                         ) m\delta z\\
                   &   - (\ddot{z}_1+\ddot{z}_2-\ddot{z}_3-\ddot{z}_4)
                         mL\delta\theta \\
                   &   - (\ddot{z}_1-\ddot{z}_2-\ddot{z}_3+\ddot{z}_4)
                         ml\delta\alpha \\
                   &   - (\delta z_1+\delta z_2
                         -\delta z_3-\delta z_4) mL\ddot{\theta}\\
                   &   -4mL^2\ddot{\theta}\delta \theta
                       -4ml^2\ddot{\alpha}\delta \alpha\\
                   &   -(\delta z_1-\delta z_2
                       -\delta z_3+\delta z_4) ml\ddot{\alpha}\\
                   &   +R_1(\delta z + \delta z_1 + L\delta\theta +
                                               l\delta\alpha)\\
                   &   +R_2(\delta z + \delta z_2 + L\delta\theta -
                                               l\delta\alpha)\\
                   &   +R_3(\delta z + \delta z_3 - L\delta\theta -
                                               l\delta\alpha)\\
                   &   +R_4(\delta z + \delta z_4 - L\delta\theta + l\delta\alpha)\;.
\end{align*}

On trouve un syst\`eme d'\'equation diff\'erentielle :
\begin{align}
\label{eqv01} \ddot{z}+\frac{m(\ddot{z}_1+\ddot{z}_2+\ddot{z}_3+\ddot{z}_4)}{M+4m} & = \frac{R_1+R_2+R_3+R_4}{M+4m}-2g\;. \\
\label{eqv02} kz_1+m(\ddot{z}_1+\ddot{z}+L\ddot{\theta}+l\ddot{\alpha}) & = -2mg +
R_1 \;. \\
\label{eqv03} kz_2+m(\ddot{z}_2+\ddot{z}+L\ddot{\theta}-l\ddot{\alpha}) & = -2mg +
R_2 \;. \\
\label{eqv04} kz_3+m(\ddot{z}_3+\ddot{z}-L\ddot{\theta}-l\ddot{\alpha}) & = -2mg +
R_3 \;. \\
\label{eqv05} kz_4+m(\ddot{z}_4+\ddot{z}-L\ddot{\theta}+l\ddot{\alpha}) & = -2mg +
R_4 \;. \\
\label{eqv06} m(\ddot{z}_1+\ddot{z}_2-\ddot{z}_3-\ddot{z}_4+4L\ddot{\theta})+\frac{I_{\theta}\ddot{\theta}}{L} 
& = R_1+R_2-R_3-R_4\;.\\
\label{eqv07}m(\ddot{z}_1-\ddot{z}_2-\ddot{z}_3+\ddot{z}_4+4l\ddot{\alpha})+\frac{I_{\alpha} \ddot{\alpha}}{l}
& = R_1-R_2-R_3+R_4 \;.
\end{align}

En faisant (\ref{eqv02}) plus (\ref{eqv03}) moins
(\ref{eqv04}) moins (\ref{eqv05}) moins (\ref{eqv06}), on obtient~:
\begin{align}
\label{eqv08} I_{\theta}\ddot{\theta} = Lk(z_1+z_2-z_3-z_4)\;.
\end{align}

En faisant (\ref{eqv02}) plus (\ref{eqv06}) moins
(\ref{eqv03}) moins (\ref{eqv04}) moins (\ref{eqv07}), on obtient~:
\begin{align}
\label{eqv09} I_{\alpha}\ddot{\alpha} = lk(z_1-z_2-z_3+z_4)\;.
\end{align}

En faisant (\ref{eqv01}) moins (\ref{eqv02}) moins (\ref{eqv03}) moins
(\ref{eqv04}) moins (\ref{eqv05}), on obtient~:
\begin{align}
\label{eqv10} M\ddot{z}=-2Mg+k(z_1+z_2+z_3+z_4)\;.
\end{align}

%En faisant (\ref{eqv04}) moins (\ref{eqv05}) on obtient~:
%$$\ddot{z}_3-\ddot{z}_4=\frac{R_3-R_4+k(z_4-z_3)}{m}+2l\ddot{\alpha}\;.$$

Soit :
$$F_1=\frac{k(z_1+z_2+z_3+z_4)}{M}\;,$$
$$F_2=\frac{L^2k(z_1+z_2-z_3-z_4)}{I_{\theta}}\;,$$
$$F_3=\frac{l^2k(z_1-z_2-z_3+z_4)}{I_{\alpha}}$$

En utilisant les \'equations (\ref{eqv08}), (\ref{eqv09}) et
(\ref{eqv10}); (\ref{eqv02})
s'\'ecrit :
$$\ddot{z}_1 =
\frac{R_1-kz_1}{m}-F_1-F_2-F_3\;.$$

De m\^eme, on trouve :
\begin{align*}
\ddot{z}_2 &=
\frac{R_2-kz_2}{m}-F_1-F_2+F_3\;,\\
\ddot{z}_3 &=
\frac{R_3-kz_3}{m}-F_{1}+F_2+F_3\;,\\
\ddot{z}_4 &=
\frac{R_4-kz_4}{m}-F_1+F_2-F_3\;.
\end{align*}

\subsubsection{Discr\'etrisation des \'equations}

Pour calculer les trajectoires des corps, nous pouvons approximer
les \'equations diff\'erentielles par des
\'equations r\'ecurrentes, o\`u $h$ d\'esigne le pas de discr\'etisation en temps~:

\begin{align*}
z_1(t+h) &= 2z_1(t)-z_1(t-h)+h^2\left
(\frac{R_1-kz_1(t)}{m}-F_1(t)-F_2(t)-F_3(t) \right ) \;,\\
z_2(t+h) &= 2z_2(t)-z_2(t-h)+h^2\left
(\frac{R_2-kz_2}{m}-F_1(t)-F_2(t)+F_3(t)  \right )\;,\\
z_3(t+h) &= 2z_3(t)-z_3(t-h)+h^2\left
(\frac{R_3-kz_3(t)}{m}-F_1(t)+F_2(t)+F_3(t)  \right )\;,\\
z_4(t+h) &= 2z_4(t)-z_4(t-h)+h^2\left
(\frac{R_4-kz_4(t)}{m}-F_1(t)+F_2(t)-F_3(t)  \right )\;,\\
z(t+h) &= 2z(t)-z(t-h)+h^2\left (
F_1(t)-2g\right )\;,\\
\alpha (t+h) &= 2\alpha (t)-\alpha (t-h)+ \frac{h^2 F_3(t)}{l}\;,\\
\theta (t+h) &= 2\theta (t)-\theta (t-h)+ \frac{h^2 F_2(t)}{L}\;.
\end{align*}


\subsection{Mouvement horizontal}\label{dyna4}

\subsubsection{Mouvement horizontal}

Pour mod\'eliser les d\'eplacements du v\'ehicule dans le plan
horizontal, on repr\'esente le v\'ehicule par une barre de demi
longueur $l$ et de masse ponctuelle $M$ \`a laquelle sont accroch\'ees
deux roues de masse $m$.  Dans un rep\`ere fixe $(Oxy)$ (voir figure
\ref{direction}), on note~:
\begin{itemize}
\item $x(t)$ et $y(t)$ la position du centre de gravit\'e de la voiture,
\item $\phi(t)$ l'angle de la carcasse de la carcasse avec l'axe $(Ox)$,
\item $\psi(t)$ l'angle des roues avec l'axe $(Ox)$.
\end{itemize}

\dessin{direction}{0.6}{Mod\'elisation de la voiture (vue de dessus)}{direction}

On note :
\begin{itemize}
\item $a(t)$ acc\'el\'eration donn\'ee par le joueur \`a la voiture
(la puissance du moteur),
\item $\xi(t)$ le changement de direction donn\'e par le joueur au v\'ehicule.
\end{itemize}

%L'\'energie cin\'etique de la voiture vaut alors : $$\frac{1}{2}(2m+M)(\dot{x}^2+\dot{y}^2)+ml\dot{\phi}^2\;.$$
%
%L'\'energie potentielle du moteur vaut : $$-a\cos \phi\ x - a\sin \phi\ y\;.$$
%
%L'\'energie potentielle de d\'eformation des pneus lors d'un changement de direction est don\'e par: $$-\frac{\xi^2}{2}\;.$$
%
%L'action \`a minimiser est donc :
%$$\AAA=1/2\int \left(\frac{1}{2}(2m+M)(\dot{x}^2+\dot{y}^2)+ml\dot{\phi}^2 +
%a\cos \phi\ x + a\sin \phi\ y -\frac{k(\psi-\phi)^2}{2}\right) dt $$

La vitesse de rotation du v\'ehicule autour de son centre de gravite est proportionnelle au changement de direction $\xi$ indiqu\'ee.
%Le moteur exerce une force d'intensit\'e $a$ choisie par le joueur
%dans la direction $\phi$.
On suppose que lors d'un changement de direction, l'\'elasticit\'e des pneus --- adh\'erents sur le sol --- produit un couple de rotation proportionel \`a $\xi$.

En n\'egligeant les frottements de l'air, on obtient les \'equations du mouvement en appliquant le principe fondamental de la dynamique :
$$(2m+M)\ddot{x} = a\cos\phi\;,$$
$$(2m+M)\ddot{y} = a\sin\phi\;,$$
$$\dot{\phi} = \xi\;.$$


\subsubsection{Discr\'etisation des \'equations diff\'erentielles}
On peut discr\'etiser ces \'equations diff\'erentielles~:
$$x(t+h) = 2x(t)-x(t-h)+h^2\left(\frac{a(t)\cos\phi(t)}{2m+M}\right)\,$$
$$y(t+h) = 2y(t)-y(t-h)+h^2\left(\frac{a(t)\sin\phi(t)}{2m+M}\right)\,$$
$$\phi(t+h) = \phi(t)+h\xi$$
o\`u $h$ d\'esigne le pas de discr\'etisation en temps.

\section{La circulation}\label{circulation}

\dessin{feu}{0.4}{La pomme est mure}{feurouge}

La circulation de la ville comprend, trois \'el\'ements : -- les feux
de signalisation, -- le joueur, -- les voitures dans les rues. 

Le plan de feux de chaque carrefour est simple : il est donn\'e par les
deux diagrames de la figure \ref{feux}.

\dessin{feux}{0.5}{Plan de feux}{feux}

Le joueur est une voiture particuli\`ere command\'ee au clavier ou \`a
la souris, en acc\'el\'eration et direction. Les touches DOWN et UP
donnent l'acc\'el\'eration et LEFT et RIGHT la direction.  Au joueur,
est associ\'e trois cam\'eras permettant de visualiser les trois
sc\`enes suivantes~: -- vue a\`erienne sur plusieurs blocs, -- vue de
la voiture du joueur, -- vue de la place du conducteur du joueur. La
commande F1 permet de passer de l'une \`a l'autre.  Les comportements
dynamiques des voitures d\'ependent du mod\'ele. La voiture n'est pas
contrainte \`a rester sur les routes, ce qui conduit \`a des
comportements sp\'ectaculaire lorsqu'on veut monter sur un immeuble ou
entrer et sortir du fleuve.


Les voitures circulent toujours sur la m\^eme voie, mais s'arr\^etent
aux feux. La vitesse des voitures est r\'egie par trois facteurs : --
une vitesse d\'esir\'ee de la voiture fluctuant avec le temps de fa\c
con al\'atoire, -- la vitesse du v\'ehicule pr\'ec\'edent -- la
distance au feu si le feu est au rouge. Une voiture \'etant associ\'ee
\`a un bloc, il faut g\'erer cette appartenance lors des changements
de blocs.

Le type TVoiture est une classe Delphi qui contient l'\'etat de la
voiture constitu\'e des onze param\`etres d\'efinissant l\'etat de sa
dynamique.  TVoiture descend de TObject, elle h\'erite donc de ses
m\'ethodes. A cela, s'ajoute de nouvelles proc\'edures qui
initialisent, actualisent la dynamique et dessinent le v\'ehicule.
 
A partir de TVoiture deux nouvelles classes ont \'et\'e cr\'e\'ees~:
TJoueur et TGoodies. La premi\`ere est la voiture contr\^ol\'ee par le
joueur, la deuxi\`eme est une voiture circulant dans la ville.

Les m\'ecanismes d'h\'eritages des classes de Delphi ont utilis\'es
comme la propri\'et\'e {\tt inherited}.  TJoueur poss\'ede de
nouvelles m\'ethodes~: les pro\'edures qui actualisent le son et la
cam\'era.

TGoodies est une TVoiture qui poss\`ede en plus deux pointeurs sur la
voiture pr\'ec\'edente et suivante. Des m\'ethodes permettent de faire
les changements de blocs des voitures.

Les TGoodies sont enchain\'ees sous forme de file associ\'ee \`a une
voie d'un bloc et constitue un objet TCirculation.  A chaque bloc, est
associ\'e un tableau de huit TCirculation.

\section{Le chargeur ASE}\label{ASE}
\subsection{Introduction}

Notre projet comprendra deux types d'objets : des objets statiques
comme la ville, le terrain... et des objets dynamiques comme les
voitures. Nous les afficherons gr\^ace \`a OpenGl, mais comme il est
imposible de dessiner une maison ou un v\'ehicule triangles par
triangles \`a la main (bien que OpenGl ne comprenne que \c ca),
nous devons utiliser un logiciel sp\'ecialis\'e dans la cr\'eation
3D~: Studio Max. Il permet d'exporter une sc\`ene 3D sous un format
ASCII~: c'est le format ASE (....).

Une sc\`ene de Studio Max, not\'ee $S$, est un ensemble d'objets $O_i$
tels que des plans, des tores, des th\'ei\`eres, etc. Donc $S =
\{O_0,O_1, ..., O_n\}.$ Chaque objet $O_i$ est compos\'e d'un
ensemble de triangles $t$ : $O_i = \{t_0, t_1, ..., t_m\}$.  Nous
appelons triangle, une surface (color\'ee) d\'efinie par trois sommets :
$t = (p_1,p_2,p_3)$ o\`u chaque sommet est d\'efini par trois
coordonn\'ees : $p_i =(x_i,y_i,z_i)$.  Enfin de compte une sc\`ene
n'est qu'un ensemble de triangles.

\subsection{Les objets 3D avec Studio Max}

Voici comment est repr\'esent\'e une sc\`ene compos\'ee de deux plans,
seul l'essentiel a \'et\'e pris.

\begin{verbatim}
*GEOMOBJECT {
        *NODE_NAME "Plan01"
        *MESH {
                *MESH_NUMVERTEX 4
                *MESH_NUMFACES 2
                *MESH_VERTEX_LIST {
                        *MESH_VERTEX 0 -50.0   -50.0   0.0
                        *MESH_VERTEX 1   0.0   -50.0   0.0
                        *MESH_VERTEX 2 -50.0     0.0   0.0
                        *MESH_VERTEX 3   0.0     0.0   0.0
                }
                *MESH_FACE_LIST {
                        *MESH_FACE 0:  A: 2   B: 0   C: 3
                        *MESH_FACE 1:  A: 1   B: 3   C: 0
                }
        }
}
*GEOMOBJECT {
        *NODE_NAME "Plan02"
        *MESH {
                *MESH_NUMVERTEX 4
                *MESH_NUMFACES 2
                *MESH_VERTEX_LIST {
                        *MESH_VERTEX 0  0.0    0.0   0.0
                        *MESH_VERTEX 1 45.0    0.0   0.0
                        *MESH_VERTEX 2  0.0   30.0   0.0
                        *MESH_VERTEX 3 45.0   30.0   0.0
                }
                *MESH_FACE_LIST {
                        *MESH_FACE 0:  A: 2   B: 0   C: 3
                        *MESH_FACE 1:  A: 1   B: 3   C: 0
                }
        }
}
\end{verbatim}

Chaque objet est repr\'esent\'e par le mot cl\'e {\tt GEOMOBJECT}. Son
nom (ici c'est un plan) est d\'esign\'e par {\tt NODE\_NAME}.  {\tt
  MESH\_NUMVERTEX} et {\tt MESH\_NUMFACES} sont respectivement le nombre de
sommets et de triangles de notre objet.  {\tt MESH\_VERTEX\_LIST} est
la liste des sommets. Chaque sommet est d\'efini par le mot cl\'e
{\tt MESH\_VERTEX} et quatre param\`etres : un entier et trois
r\'eels.  L'entier est le num\'ero d'indentification du sommet, les
r\'eels repr\'esentent une position dans l'espace.  Par exemple :
\begin{verbatim}
*MESH_VERTEX 0 -50.0   -50.0   0.0
\end{verbatim}
est le premier sommet de position $(-50,-50,0)$ et s'\'ecrit :
$p_0 =(-50,-50,0)$.

{\tt MESH\_FACE\_LIST} est la liste des triangles de notre objet. Chaque
triangle est d\'efini par le mot cl\'e {\tt MESH\_FACE} et sept
param\`etres. Par exemple :
\begin{verbatim}
*MESH_FACE 0:  A: 2   B: 0   C: 3
\end{verbatim}

Le $0:$ est le numero d'identification du triangle. Le {\tt A:} est le
premier sommet du triangle, le {\tt B:} le second et enfin le {\tt C:} le
troisi\`eme. Le $2$ de {\tt A: 2} est un pointeur vers le troisi\`eme
sommet de la liste des sommets. Il pointe sur~:

\begin{verbatim}
*MESH_VERTEX 2 -50.0   0.0   0.0
\end{verbatim}
Le principe est le m\^eme pour le $0$ de B: $0\;$, il pointe sur~:
\begin{verbatim}
*MESH_VERTEX 0 -50.0   -50.0   0.0
\end{verbatim}
et enfin le $3$ de C: $3\;$, pointe sur :
\begin{verbatim}
*MESH_VERTEX 3 0.0   0.0   0.0
\end{verbatim}
En d\'efinitive, nous avons : $t_0 = (p_2,p_0,p_3).$

\subsection{Les objets 3D avec OpenGl}

Une sc\`ene avec Opengl est elle aussi un ensemble de triangles.
Voici comment est repr\'esent\'ee la m\^eme sc\`ene toujours
compos\'ee de deux plans.

\begin{verbatim}
glBegin(GL_TRIANGLES);
// Premier triangle du premier plan
   glVertex3f(-50.0,0.0,0.0);
   glVertex3f(-50.0,-50.0,0.0);
   glVertex3f(0.0,0.0,0.0);

// Deuxieme triangle du premier plan
   glVertex3f(0.0,-50.0,0.0);
   glVertex3f(0.0,0.0,0.0);
   glVertex3f(-50.0,-50.0,0.0);

// Premier triangle du deuxieme plan
   glVertex3f(0.0,30.0,0.0);
   glVertex3f(0.0,0.0,0.0);
   glVertex3f(45.0,30.0,0.0);

// Deuxieme triangle du deuxieme plan
   glVertex3f(45.0,0.0,0.0);
   glVertex3f(45.0,30.0,0.0);
   glVertex3f(0.0,0.0,0.0);
glEnd;
\end{verbatim}

Le probl\`eme principal du chargeur est de stocker correctement la
sc\`ene, pour qu'elle prenne le moins de place m\'emoire.
L'utilisation de liste chain\'ee est plus astucieuse que celle d'un
tableau, car nous ne connaissons pas \`a l'avance le nombre exact
d'objets ou de triangles. On cr\'ee alors une liste de liste. La
premi\`ere contient tous les objets. Dans chaqu'une de ses cases on
stocke la liste de triangles propre \`a chaque objet.

Une remarque importante est que la ligne suivante n'est pas compl\'ete :
\begin{verbatim}
*MESH_FACE 0:  A: 2   B: 0   C: 3
\end{verbatim}
Elle s'\'ecrit, en faite :
\begin{verbatim}
*MESH_FACE 0: A: 2 B: 0 C: 3 AB: 1 BC: 0 CA: 1 *MESH_SMOOTHING 1 
 *MESH_MTLID 0
\end{verbatim}

Les valeurs de {\tt AB:}, {\tt BC:} et {\tt CA:} sont des bool\'eens repr\'esent\'es
sous la forme de $1$ ou de $0$. Ce sont des drapeaux d'arr\^ete (Edge
Flag), \`a savoir des bascules qui permetent de n'afficher que
certaines arr\^etes d'un polygone. {\tt glEdgeFlag} est une proc\'edure
OpenGl qui prend en param\`etre un bool\'een. Seules les ar\^etes sur
{\tt TRUE} seront affich\'ees. Ceci n'a d'int\'er\^et que pour un affichage
en mode fil de fer.

\subsection{Textures et mat\'eriaux avec Studio Max}

Maintenant que nous pouvons exporter n'importe qu'elle sc\`ene, nous
allons l'\'egayer en appliquant une texture \`a nos triangles. Le
placage de texture est une technique permettant d'accro\^itre le
r\'ealisme d'un rendu 3D. Il consiste \`a coller une image sur un
objet 3D \`a la mani\`ere d'une tapisserie.

Voici comment est repr\'esent\'e une sc\`ene compos\'ee de d'un seul
plan \`a quatre triangles et d'une texture appel\'ee camouflage.bmp.
Elle se trouve dans le chemin~: {\tt
  "C\!\!:$\backslash$camouflage.bmp"} (seul l'essentiel a \'et\'e
pris).

\begin{verbatim}
*MATERIAL_LIST {
        *MATERIAL_COUNT 1
        *MATERIAL 0 {
                *MATERIAL_NAME "Materiau #1"
                *MATERIAL_AMBIENT 0.2 0.1 0.1
                *MATERIAL_DIFFUSE 0.5 0.2 0.2
                *MATERIAL_SPECULAR 0.9 0.9 0.9
                *MATERIAL_SHINE 0.2
                *MATERIAL_SHINESTRENGTH 0.05
                *MATERIAL_TRANSPARENCY 0.0
                *MAP_DIFFUSE {
                        *MAP_NAME "Texture01"
                        *MAP_CLASS "Bitmap"
                        *BITMAP "C:\camouflage.bmp"
                        *UVW_U_OFFSET 0.0
                        *UVW_V_OFFSET 0.0
                }
        }
}
*GEOMOBJECT {
        *NODE_NAME "Plan01"
        *MESH {
                *MESH_NUMTVERTEX 8
                *MESH_TVERTLIST {
                        *MESH_TVERT 0 0.0 0.0 0.0
                        *MESH_TVERT 1 1.0 0.0 0.0
                        *MESH_TVERT 2 0.0 0.0 0.0
                        *MESH_TVERT 3 1.0 0.0 0.0
                        *MESH_TVERT 4 0.0 0.0 0.0
                        *MESH_TVERT 5 1.0 0.0 0.0
                        *MESH_TVERT 6 0.0 1.0 0.0
                        *MESH_TVERT 7 1.0 1.0 0.0
                }
                *MESH_NUMTVFACES 2
                *MESH_TFACELIST {
                        *MESH_TFACE 0 6 4 7
                        *MESH_TFACE 1 5 7 4
                }
        }
        *MATERIAL_REF 0
}
\end{verbatim}

Pour cr\'eer une texture sur un triangle, Studio Max d\'efinit d'abord
les caract\'eristiques du mat\'eriau, ensuite il calcule les sommets
et dans un dernier temps, il d\'efinit le triangle gr\^ace \`a la
liste de sommet et au num\'ero d'identification du bitmap.

Notre liste de mat\'eriau est appel\'ee par le mot cl\'e :
{\tt MATERIAL\_LIST}. Ici, nous en n'avons qu'un seul ({\tt MATERIAL\_COUNT 1}),
appel\'e Materiau $01$.  Les mots cl\'es suivants : de
{\tt MATERIAL\_AMBIENT} \`a {\tt MATERIAL\_TRAN\-SPARENCY} ne sont utiles que pour
la r\'eflection de la lumi\`ere (ambiante, sp\'eculaire, diffuse) et
la transparence de l'objet.

Mais ce qui nous int\'eresse le plus est le chemin du bitmap ({\tt
  BITMAP "C\!\!:$\backslash$camouflage.bmp"}) et le d\'ecalage de la texture sur
l'axe $u$ et $v$ (d\'efinis respectivement par {\tt UVW\_U\_OFFSET} et
{\tt UVW\_V\_OFFSET}). En effet, pour positionner une texture sur un
polygone, on parle de 'UV mapping'. Le principe est simple~: on
affecte un syst\`eme de coordonn\'ees $(u,v)$ \`a la texture suivant
l'illustration. Positionner la texture consiste \`a affecter \`a
chaque sommet d'un polygone la coordonn\'ee de la texture en ce point.
Dans notre cas, notre plan se divise en triangles, les coordonn\'ees
aux sommets des points des faces sont : $(0,0), (1,0), (1,1)$ et
$(0,0), (0,1), (1,1)$.

\dessin{texture}{0.8}{Coordonn\'ees d'un mat\'eriau sur un objet carr\'e}{texture}

Une fois que la d\'efinition de notre texture finie, nous allons
cr\'eer notre liste de triangles de texture. Le principe est
exactement le m\^eme que dans la section pr\'ec\'edente. Chaque
texture est d\'efinie par le mot cl\'e {\tt MESH\_TVERT}, un entier et trois
r\'eels.  L'entier est le num\'ero d' indentification de la texture, les
r\'eels d\'efinissent la position dans l'espace. Par exemple :
\begin{verbatim}
*MESH_TVERT 0 0.0 0.0 0.0
\end{verbatim}
est le premier sommet $p_0 = (0,0,0).$

{\tt MESH\_TFACELIST} est notre listre de texture. Chaque texture est
appel\'ee par {\tt MESH\_FACE\_LIST} et quatre param\`etres (quatre entiers)
: un numero d'indentification et trois pointeurs vers un {\tt MESH\_TVERT}.
Par exemple : $t_0 = (p_6,p_4,p_7).$ est appel\'e par :
\begin{verbatim}
*MESH_TFACE 0 6 4 7
\end{verbatim}

Pour l'instant, nous ne savons pas quelle texture utilis\'ee (bien
qu'il n'y en ait qu'une ici). C'est {\tt MATERIAL\_REF} qui nous l'indique.
L'entier qu'il prend en param\`etre est un pointeur sur le num\'ero du
mat\'eriau. Par exemple {\tt MATERIAL\_REF 0} pointe sur {\tt MATERIAL 0}.

\subsection{Les textures d'OpenGl}

Le code est en Pascal. Chaque texture doit obligatoirement \^etre
charg\'ee par la fonction {\tt ChargerTexture} (il existe des
proc\'edures plus complexes, mais celle-ci est la plus simple). Une
texture est d\'efinie par un entier OpenGL, ou Gluint, c'est en
quelque sorte son nom. La proc\'edure {\tt ChargerTexture} prend deux
param\` etre : le num\'ero de la texture et son chemin dans
l'arborescence du syst\`eme d'exploitation (ici Windows).

\begin{verbatim}
procedure ChargerTexture(NumeroTexture : gluint;
                         Chemin : string);
var texture1: PTAUX_RGBImageRec;
begin
//Chargement de la texture avec glaux
  texture1 := auxDIBImageLoadA(Pchar(Chemin));

//Initialisation de la texture
  glGenTextures(1, NumeroTexture);
  glBindTexture(GL_TEXTURE_2D, NumeroTexture);
  glTexParameteri(GL_TEXTURE_2D,GL_TEXTURE_MAG_FILTER,
                  GL_LINEAR);
  glTexParameteri(GL_TEXTURE_2D,GL_TEXTURE_MIN_FILTER,
                  GL_LINEAR);
  glTexImage2D(GL_TEXTURE_2D, 0, 3, texture1^.sizeX,
               texture1^.sizeY,0,GL_RGB,
               GL_UNSIGNED_BYTE, texture1^.data);
end;

procedure AfficherUnTrangle(NumeroTexture : gluint);
begin
//Plaquage de la texture 'NumeroTexture'
  glBindTexture(GL_TEXTURE_2D,NumeroTexture);

//Affichage
  glBegin(GL_TRIANGLES);
     glTexCoord3f(1.0,0.0,0.0); glVertex3f(50.0,0.0,0.0);
     glTexCoord3f(1.0,1.0,0.0); glVertex3f(50.0,10.0,0.0);
     glTexCoord3f(0.0,0.0,0.0); glVertex3f(0.0,0.0,0.0);
  glEnd;
end;
\end{verbatim}

Par rapport \`a la section 'Les objets 3D avec OpenGL', dans la
proc\'edure {\tt AfficherUnTriangle} on a ajout\'e \`a chaque ligne un
{\tt glTexCoord3f}.  Il permet d'indiquer les sommets de la texture.
{\tt glBindTexture} demande \`a OpenGL d'appliquer la texture sur le
triangle (sinon le triangle prend une couleur monochrome par
d\'efaut).


\subsection{Cam\'era cible}

Voici comment est repr\'esent\'ee une cam\'era cible sous Studio
Max (seul l'essentiel a \'et\'e pris) :

\begin{verbatim}
*CAMERAOBJECT {
        *NODE_NAME "Camera01"
        *CAMERA_TYPE Target
        *NODE_TM {
                *NODE_NAME "Camera01"
                *TM_POS 100.0 100.0 100.0
        }
        *NODE_TM {
                *NODE_NAME "Camera01.Target"
                *TM_POS 1.0 1.0 1.0
       }
}
\end{verbatim}

Voici comment Opengl g\'ere une cam\'era :
\begin{verbatim}
  gluLookAt(100.0,100.0,100.0,
            1.0,1.0,1.0,
            0.0,1.0,0.0)
\end{verbatim}

Une cam\'era cible est d\'efinie par une position $p$ dans l'espace
$p=(p_x,p_y,p_z)$, un point qu'elle cible $c=(c_x,c_y,c_z)$ et un
vecteur d'orientation (normale) $o=(o_x,o_y,o_z)$ car touner une
cam\'era d'un angle quelconque sur l'axe du vecteur position--cible
modifie l'orientatio de l'image (par exemple, on voit diff\'erament
notre environnement selon qu'on a la t\^ete \`a l'envers ou pas).
Dans notre exemple, nous avons : $p=(100.0,100.0,100.0)$,
$c=(1.0,1.0,1.0)$ et $o=(0.0,1.0,0.0)$.

\section{Effets sp\'eciaux}
\subsection{Les liste d'affichage d'OGL  \cite{LM33}} 
Les listes d'affichage contribuent \`a l'am\'elioration des perfomances
graphiques parce qu'elles permettent de stocker des commandes OpenGL
pour une utilisation future. L'utilit\'e la plus \'evidente est
l'enregistrement d'objets d'utilisation fr\'equente. Par exemple pour dessiner une
salle de classe, il suffit d'enregistrer l'objet chaise dans une liste
d'affichage, que l'on appelle autant de fois que l'on a de chaises
dans la salle.  Une fois la liste d'affichage cr\'e\'ee, il n'est plus
possible de la modifier, sinon les performances s'en trouveraient
affect\'ees par le sondage de la liste et la gestion de la m\'emoire
(fragmentation de la m\'emoire). Elles permettent notamment
d'optimiser les op\'erations matricielles, les lumi\`eres, mat\'eriaux
et les textures. Voici, comment elles fonctionnent.

\begin{verbatim}
function GlGenLists(n : GLsizei) : Gluint;
\end{verbatim}
avec $n$ le nombre de listes que l'on souhaite cr\'eer. 
La fonction renvoie un bloc de $n$ identifiants.
\begin{verbatim} 
procedure glNewList(liste Gluint, mode GLenum);
\end{verbatim}
avec liste doit \^etre un entier retourn\'e par glGenLists();. mode
peut prendre les valeurs {\tt GL\_COMPILE} et {\tt
  GL\_COMPILE\_AND\_EXECUTE}.  La derni\`ere enregistre et ex\'ecute
immediatement. Si on utilise {\tt GL\_COMPILE}, les instructions ne
sont pas ex\'ecut\'ees lors de l'enregistrement.

Les instructions OpenGl qui suivent l'appel de {\tt glNewList} sont
m\'emoris\'ees dans la liste d'affichage jus'qu\`a l'instruction qui
marque la fin de l'enregistrement.
\begin{verbatim}
procedure glEndList();
\end{verbatim}

Pour ex\'ecuter une liste d'affichage, il suffit d'appeler : 
\begin{verbatim}
procedure glCallList(liste : Gluint);
\end{verbatim}
o\`u liste est l'inditifiant de la liste. Il est \'egalement possible
d'\'effacer une liste d'affichage avec :
\begin{verbatim}
procedure glDeleteLists(liste : Gluint; n : GLsizei);
\end{verbatim}
o\`u $n$ d\'esigne le nombre de liste \`a \'effacer (le m\^eme nombre de
liste r\'eserv\'ees par {\tt glGenLists()}).

\subsection{Les lumi\`eres  \cite{LM30}}\label{Lumieres}
\subsubsection{Sources lumineuses}
La lumi\`ere \'emise par une source est form\'ee de trois composantes.
La plus importante, la composante diffuse, est r\'efl\'echie par un
objet dans toutes les directions. La composante sp\'eculaire
correspond \`a la lumi\`ere qui est r\'efl\'echie dans une direction
privil\'egi\'ee (et qui est donc \`a l'origine de l'effet de
brillance). La composante ambiante d'une sc\`ene est une lumi\`ere non
directionnelle, que l'on peut consid\'erer comme issue des multiples
r\'eflexions de rayons lumineux.

Pour des raisons d'efficacit\'e, OpenGL ne calcule pas la couleur de
chaque pixel d'un polygone :  -- soit il remplit chaque polygone avec une
couleur unie (model de remplissage \emph{Flat}),  -- soit il utilise un
algorithme de Gouraud (mode \emph{Smooth})\footnote{Le principe de l'algorithme de
Gouraud est le suivant : pour un polygone donn\'e, la couleur de chacun des sommets
est calcul\'ee et le polygone est rempli avec un d\'egrad\'e entre ces
differentes couleurs}. Pour calculer correctement la r\'eflexion des
rayons lumineux en un point OpenGL a besoin de conna\^itre la
perpendiculaire \`a la surface de l'objet au point donn\'e (c'est la
normale).

\subsubsection{Impl\'ementation avec OpenGL}
La phase d'initialisation de l'\'eclairage commence par la sp\'eci\-fication
du mode de remplissage des polygones avec :

\begin{verbatim}
  glShadeModel(GL_SMOOTH);
\end{verbatim}

Ensuite, on indique \`a OpenGL qu'on souhaite utiliser le calcul
d'\'eclairage, en activant la variable d'\'etat {\tt GL\_LIGHTING}~:

\begin{verbatim}
  glEnable(GL_LIGHTING);
\end{verbatim}

OpenGL permet d'utiliser jusqu'\`a huit sources de lumi\`ere. Ces huit
lampes sont index\'ees par les constantes {\tt GL\_LIGHT0} \`a {\tt GL\_LIGHT7}. Il faut
activer chacunes des sources qu'on souhaite utiliser (une dans notre
cas)~:

\begin{verbatim}
  glEnable(GL_LIGHT0);
\end{verbatim}

Vient ensuite le param\`etrage des lampes. Il se fait avec une unique
fonction, {\tt glLightfv()}, dont le prototype est le suivant :
\begin{verbatim}
  procedure glLightfv( lampe    : Glenum; 
                       nomparam : Glenum; 
                       param    : GLType );
\end{verbatim}
 \emph{lampe} d\'esigne la lampe dont on veut modifier une propri\'et\'e. 
 \emph{nomparam} est le nom du param\`etre \`a modifier. Il s'agit d'une des dix
propri\'et\'es de source lumineuses
( {\tt GL\_DIFFUSE}, {\tt GL\_AMPBIANTE}, {\tt GL\_SPECULAR}, {\tt GL\_POSITION}, etc ).   \emph{param}
d\'esigne la valeur \`a affecter au param\`etre choisi (les param\`etres sont
pass\'es sous forme de tableaux).

\subsection{Param\`etres de mat\'eriaux}
Dans la r\'ealit\'e, un rayon lumineux est constitu\'e d'un ensemble
d'ondes de longueurs diff\'erentes. A chaque longueur d'onde correspond
une couleur. Un objet frapp\'e par un rayon lumineux va absorber
certaines longueurs d'onde et r\'efl\'echir les autres selon  les
caract\'eristiques de ce mat\'eriau.

Lorsqu'un polygone est d\'ecrit, il
se voit affecter le mat\'eriau courant.  La modification du mat\'eriau
courant se fait avec la fonction {\tt GlMaterialfv()} :
\begin{verbatim}
  procedure glMaterialfv( face     : Glenum; 
                          nomparam : Glenum;
                          param   : GlType );
\end{verbatim}
 \emph{face} indique la face (avant ou arri\`ere) dont on souhaite modifier
les param\`etres.  \emph{nomparam} d\'esigne la propri\'et\'e qu'on souhaite
changer, et \emph{param} est un tableau contenant la nouvelle valeur \`a
affecter \`a  \emph{nomparam}. Les valeurs de  \emph{nomparam} possibles sont~:
{\tt GL\_AMBIENT}, {\tt GL\_DIFFUSE}, {\tt GL\_SPECULAR}, {\tt GL\_SHININESS} (i.e. coefficient de
brillance).

\subsection{La transparence}\label{blend}
\subsubsection{Fusionnement des couleurs} 
Le quatri\`eme param\`etre de la commande {\tt glColor4f}, appel\'e
\emph{alpha}, sert \`a sp\'ecifier la couleur d'un objet 
et est utilis\'e pour sp\'ecifier le taux de transparence
d'un objet.

Plus la valeur alpha sera forte (le maximum \'etant de $1$) plus la
primitive sera opaque.  Inversement, plus la valeur sera faible (le
minimum est $0$) plus la transparence sera forte.

Si on prend un cube ayant la valeur alpha \`a $1$ et que l'on place
devant une sph\`ere avec une valeur alpha de $0.15$, le cube ne
laissera passer aucune couleur (mais ce n'est pas grave car il est
situ\'e \`a l'arri\`ere de la sph\`ere), par contre, la sph\`ere
laisse passer $15$ pourcent de transparence. Logiquement, si on
regarde la sph\`ere, on devrait voir le cube. Si le blending n'est pas
activ\'e et que les deux objets sont dessin\'es l'un par dessus
l'autre, OpenGL les dessinera comme \'etant deux objets opaques. En
effet lorsque le deuxi\`eme objet est affich\'e, OpenGL \'ecrase
chaque valeur chromatique du cube pour les remplacer par celle de la
sph\`ere.  Par contre, si le blending est activ\'e, alors OpenGL
conserve celles du cube et les accouple \`a celles de la sph\`ere.
 
La sph\`ere joue le r\^ole de la \emph{source}, car ses valeurs
chromatiques seront affect\'ees \`a ceux du cube. Le cube joue le
r\^ole de la \emph{destination} car il est le r\'esultat de
l'accouplement de ses valeurs chromatiques avec celles de la sph\`ere.

\subsubsection{Impl\'ementation}
L'activation de la transparence, se fait gr\^ace \`a la commande~:
{\tt glEnable(GL\_BLENDING)} et pour la d\'esactiver :
{\tt glDisable( GL\_BLENDING)}.

Pour voir les diff\'erents types d'accouplement entre la source et la
destination, conf\'ere le livre de
ref\'erence OpenGL~:
\begin{verbatim}
  glBlendFunc(Source : GLenum; Destination :  GLenum);
\end{verbatim}

Prenons un exemple (le blending est activ\'e):
\begin{verbatim}
  glBlendFunc(GL_ONE, GL_ZERO);
\end{verbatim}

Si l'on dessine une primitive en utilisant le blending de
cette fa\c con, cela n'aura aucun effet de transparence (on
obtiendra les m\^emes r\'esultats que si le blending \'etait
d\'esactiv\'e). Parce que les valeur chromatiques
de la source sont multipli\'ees par $1$, ce qui ne change
rien aux valeurs chromatiques de la source. Par contre les
valeurs chromatiques de la destination sont \'ecras\'ees car elles sont
multipli\'ees par $0$.

\subsection{La brume}\label{fog}
\subsubsection{Utilit\'e}
La brume (ou fog en anglais) est utile pour apporter du r\'ealisme aux
sc\`enes 3D (par exemple pour dessiner des montagnes).  Sur des gros
programmes en 3D, la brume peut am\'eliorer les performances en
\'eliminant les objets dont la distance est trop grande pour \^etre
per\c cue par la cam\'era.

Pour notre projet, il permet d'\'eviter l'apparition brutale des 
immeubles dans le c\^one de vision lorsqu'on ne les visualise
 pas tous  (voir section \ref{frustum}).


\subsubsection{Impl\'ementation}
Voici l'impl\'ementation de la brume avec OpenGL :
\begin{verbatim}
  (1) FogCouleur : Array[0..3] of GLFloat = (0.5,0.5,0.5,1.0); 
  (2) glFogf(GL_FOG_MODE, GL_LINEAR);
  (3) glFogf(GL_FOG_DENSITY, 0.35); 
  (4) glFogi(GL_FOG_START, 3.0); 
  (5) glFogi(GL_FOG_END, 30.0); 
  (6) glFogfv(GL_FOG_COLOR, @FogCouleur); 
  (7) glEnable(GL_FOG);
\end{verbatim}

A la ligne (1), le tableau FogCouleur contient la couleur de la brume
gr\^ace au quadruplet (R,G,B,A) qui d\'esigne respectivement la
couleur rouge, verte, bleue et alpha (opacit\'e).  

Ligne (2), on
d\'etermine le type de brume que l'on souhaite avoir~: plus ou moins
\'epaisse selon la distance (lin\'eaire ou exponentielle) :
GL\_LINEAR, GL\_EXP ou GL\_EXP2.  

Ligne (3), on d\'etermine la
densit\'e de la brume. La plus petite valeur est $0$ et la la plus
\'elev\'ee est $1$.  

Ligne (4), on d\'etermine la position o\`u la
brume commence \`a faire effet.  

Ligne (5), on d\'etermine la position
o\`u la brume perd son effet et par cons\'equent o\`u OpenGL arr\^ete
de dessiner les primitives car elles sont ``perdues dans la brume''.

Ligne (6), on affecte une couleur \`a la brume, ici la couleur est
contenue dans notre tableau FogCouleur.  

Enfin, \`a la ligne (7), on
active la brume.
\subsection{Moteur de particules}

Dans le troisi\`eme rapport, nous avons fait de tr\`es grossi\`eres
et impardonnables erreurs de physique. Nous n'avons pas eu le temps de
les corriger. 

\subsection{Le terrain}\label{terrain}
\subsubsection{Le principe des heightfields}

Le principe de champs de hauteur est relativement simple \`a
comprendre. Dans un premier temps, nous allons generer un maillage
carre dans le plan XY.
 
Le maillage est illustr\'e figure \ref{grille}. Ce n'est ni plus ni moins qu'un
ensemble de polygones carr\'es (appel\'es patch pour le distinguer du
maillage complet) coll\'es les uns aux autres. L'op\'eration se
r\'ealise par deux boucles imbriqu\'ees. Une fois le maillage cr\'ee,
il n'y a plus qu'\`a affecter aux sommets de chaque patch une
coordonn\'ee en Z. En fonction de la position du sommet consid\'er\'e dans
le maillage, on \'etablit une correspondance avec le pixel de l'image en
niveau de gris. Si le pixel correspondant est noir, l'\'el\'evation du
sommet sera minimale alors que s'il est blanc, la hauteur sera
maximale.

\subsubsection{Des polygones qui d\'eg\`enerent  \cite{LM33}}
En utilisant le principe pr\'ec\' edant, nous
sommes confront\'e a un probl\`eme : nous allons cr\'eer des polygones
deg\'en\'er\'es. OpenGL d\'efinit strictement les propri\'et\'es que doit
respecter un polygone pour obtenir un rendu correct. Les polygones
doivent etre convexes (c'est \`a dire que toute ligne joignant deux
points quelconques du ploygone doit \^etre enti\'erement comprise dans le
polygone), non auto-intersectants et tous les points doivent \^etre sur
le m\^eme plan. Or, cette derni\`ere condition n'est pas verifi\'ee
lorsqu'on gen\`ere le terrain puisque les valeurs d'\'el\'evation des
sommets sont issus d'une image, et il y a peu de chance que les quatre
sommets de chaque patch soient dans un m\^eme plan. Dans le cas
d'affichage en mode fil de fer, la d\'eg\'eneration des polygones ne pose
pas probl\`eme. En revanche, si on \'eclaire la sc\`ene, le rendu des faces
en mode plein sera faux. Autant prendre les devants et afficher
quelque chose de correct.

La solution au probl\`eme de d\'eg\'en\'eration est simple : il suffit de
d\'ecomposer chacun de nos patchs en deux triangles. Notre maillage
global ressemble donc \`a la figure \ref{grille}.  L'arrete transversale est
disgracieuse. Nous allons nous en d\'ebarasser gr\^ace en utilisant la
fonction glEdgeFlag d'OpenGL qui permet de n'afficher que certaines
arr\^etes d'un polygone. Le prototype de glEdgeFlag est le suivant :
\begin{verbatim}
procedure glEdgeFlag(valeur : boolean);
\end{verbatim}
valeur peut valoir TRUE ou FALSE. les arr\^etes dont le premier sommet est
d\'efini lorsque le drapeau est sur TRUE seront affich\'ees. En revanche,
si le drapeau est sur FALSE, l'arr\^ete n'est pas affich\'ee. Bien sur
ceci n'a d'int\'erer que lors d'un affichage en mode de fer.

\subsubsection{Le format RAW}
C'est avec des textures au format RAW que l'on va calculer la hauteur
de chaque sommet du patch, car c'est un format permettant de lire une
image comme un tableau de nombres entiers compris entre 0 et 255. Il
est donc tr\`es utile pour passer d'une image \`a un tableau de calcul
et r\'eciproquement. Pour ce faire nous avons besoin d'utiliser la
fonction blockread qui lit un ou plusieurs enregistrements d'un
fichier ouvert et les place dans une variable.  Voici comment
fonctionne cette fonction :
\begin{verbatim}
procedure BlockRead( var Fichier : File; 
                     var Tampon; 
                     compte : Integer 
                    [;var Transfere : Integer]
                   );
\end{verbatim}
BlockRead lit au moins Compte enregistrements \`a partir du fichier
Fichier et les transf\`ere en m\'emoire en partant de l'octet occup\'e
par Tampon.  Le nombre r\'eel d'enregistrements entiers lus (inf�rieur
ou \'egal � Compte) est renvoy\'e dans Transfere.  On utilisera aussi
la fonction SizeOf(X) qui renvoie le nombre d'octets occup\'es par une
variable ou un type (ici X).

En un mot, pour calculer la hauteur de

On cr\'eait la procedure LoadRawFile qui utilise le tableau de byte appel\'e
Relief et une constante de type chaine, Chemin, qui indique le chemin de 
la texture.
\begin{verbatim}      
CONST Chaine = 'C:\MaTexture.raw'; 
VAR {parametre global}
    Relief : array [0..1023,0..1023] of byte;

procedure LoadRawFile();
begin    
{ Ouverture du fichier RAW,  Attention : on supose que le 
 fichier existe
}
    AssignFile(Fichier, chemin);
    BlockRead(F,image,sizeof(image));
    CloseFile(Fichier);
end; {LoadRawFile} 
\end{verbatim}

\subsubsection{La cr\'eation du terrain \cite{LM33}}
Le terrain est de dimension fixe : il sagit d'un carr\'e dans le plan
XY dont les extr\'emit\'es sont les points
$(-1,-1,),(-1,1),(1,-1),(1,)$. La densit\'e du maillage on utilise une
constante NB\_SUBDIVISION qui repr\'esente le nombre de subdivisions
sur chaque axe. Les tableaux P1,P2,P3 et P4 ont chacun quatre cases qui
contiennent une alititude.

\dessin{terr}{0.8}{Maillage avec les faces transversales}{grille}

\begin{verbatim}
CONST NB_SUBDIVISIO = 4;
for i := 0 to NB_SUBDIVISION do
begin
    for j := 0 to NB_SUBDIVISION do
    begin
       P1[0] := -1+i*pas; P1[1] := -1+j*pas;
       P1[2] := Relief(i,j);
        
       P2[0] := -1+(i+1)*pas; P2[1] := -1+j*pas;
       P2[2] :=  Relief(i+1,j);

       P3[0] := -1+(i+1)*pas; P3[1] := -1+(j+1)*pas;
       P3[2] := Relief(i+1,j+1);

       P4[0] := -1+i*pas; P4[1] := -1+(j+1)*pas;
       P4[2] := Relief(i,j+1);
    end;
end;
\end{verbatim}

\section{Utilitaires}

\subsection{Importation de voiture}

Si un joueur veut d\'efinir lui-m\^eme sa propre voiture dans 3DS. Il peut
facilement l'importer dans le jeu de la fa\c con suivante~:
\begin{itemize}
\item cr\'eer un nouveau dossier avec le nom de la voiture;
\item dessiner une carcasse de voiture avec 3D Studio Max et l'exporter sous 
\emph{Carcasse.ase} dans le dossier;
\item dessiner \'egalement une roue (la gauche) et l'exporter sous  \emph{Roue.ase}
toujours dans le m\^eme dossier;
\item prendre une photo de la voiture \emph{photo.jpg};
\item cr\'eer un fichier \emph{info.txt} et mettre les param\`etres de la voiture 
(poids de la roue, de la carcasse, raideur des ressort, position des roues etc);
\item lancer le programme, s\'electionner la nouvelle voiture et jouer;
\item si la voiture ne plait pas, alors la supprimer du dossier.
\end{itemize}

Remarques :
\begin{itemize}
\item on peut ajouter ou supprimer une nouvelle voiture et modifier
  ses param\`etres dans son fichier  \emph{info.txt} en cours de jeu;
\item modifier les param\`etres de la voiture, sans comprendre ce que
  l'on fait, peut conduire \`a des probl\`emes num\'eriques; par
  exemple, si la r\'eaction du sol est trop forte ou le pas en temps
  est trop grand des instabilit\'es num\'eriques peuvent apparaitre.
\end{itemize}

La d\'efinition d'une voiture n\'ec\'essitant plusieurs fichiers
(g\'eom\'etrie de la roue, g\'eom\'etrie de la caracasse, param\`etres
m\'ecaniques et photo), Delphi ne possedant pas de proc\'edures
permettant de stocker dans un tableau les noms des dossiers contenus
dans un dossiers, il a \'et\'e n\'ec\'essaire de faire une proc\'dure
remplissant cette fonction :
\begin{verbatim}
    (1)var sr : TsearchRec;
    (2)      Liste : Tliste; 

    (4)   FindFirst(GetCurrentDir+'*.*', faDirectory, sr);
    (5)   while ((FindNext(sr) = 0) and (Liste.long < 10) do
          begin
              Liste.long := Liste.long+1;
    (7)       Liste.elt[Liste.long].Nom := sr.name;
          end;
\end{verbatim}

Le type TSearchRec est rempli lors d'un appel aux
fonctions FindFirst ou FindNext. En particulier, si un fichier est trouv\'e, le 
champ  \emph{name} de TSearchRec est rempli par le nom du fichier.

Une Tliste est un enregistrement qui poss\'ede deux
champs. Le premier  \emph{elt} est un tableau de strings
et le deuxi\`eme  \emph{long} est la position du dernier \'el\'ement.

La fonction FindFirst, 
\begin{verbatim}
  function FindFirst( const path : string ; 
                      attr       : Integer; 
                      var F      : TSearchRec ): Integer;
\end{verbatim}
recherche dans le r\'epertoire sp\'ecifi\'e par
 \emph{path} le premier fichier qui correspond au nom de fichier
sp\'ecifi\'e par  \emph{path} et aux attributs sp\'ecifi\'es par le
param\`etre  \emph{attr}. Le r\'esultat est renvoy\'e dans le param\`etre
 \emph{F}. FindFirst renvoie $0$ si un fichier a \'et\'e localis\'e avec
succ\`es; sinon, elle renvoie un code d'erreur Windows.

Le param\`etre constant  \emph{path} correspond \`a un r\'epertoire et un
masque de fichier qui peut inclure des caract\`eres g\'en\'eriques. Par
exemple, \emph{c:$\backslash$test$\backslash$*.*} indique tous les
fichiers du r\'epertoire \emph{C:$\backslash$TEST}).

Outre les fichiers normaux, le param\`etre  \emph{attr} indique les fichiers
sp\'eciaux \`a prendre en compte. Par exemple : faReadOnly pour les
fichiers en lecture seule ou faDirectory pour les fichiers
r\'epertoire, etc. Il est a not\'e que la fonction GetCurrentDir
retourne le chemin courant. Par d\'efaut, elle retourne le chemin o\`u
se trouve l'ex\'ecutable.

La fonction FindNext,
\begin{verbatim}
  function FindNext( var F : TSearchRec ): Integer;
\end{verbatim}
renvoie l'entr\'ee suivante correspondant au nom de fichier et
\`a l'ensemble des attributs pr\'ealablement d\'efinis lors d'un appel \`a la
fonction FindFirst. L'enregistrement de recherche doit \^etre identique
\`a celui transmis \`a la fonction FindFirst. La valeur renvoy\'ee est $0$
si l'ex\'ecution de la fonction \`a r\'eussi. Sinon, cette valeur est un code
d'erreur Windows.

A la ligne (6), on ajoute le nom du dossier dans la derni\`ere case du
tableau.  Remarque : le premier \'el\'ement du tableau est la chaine
 ''..'', s'il existe un dossier parent.

\subsection{Courbes de Bezier}
Les courbes de Bezier pourraient servir pour raccorder les routes de
fa\c{c}on plus lisse. Bien que dans l'impl\'ementation actuelle cela
n'a pas \'et\'e faite --- pour pouvoir mieux appr\'ecier le
fonctionnement des suspensions --- donnons une introduction aux
courbes de Bezier.

Les courbes de Bezier sont des courbes \`a fonctions param\'etriques
$x(t)$ et $y(t)$ de degr\'e trois, c'est \`a dire qu'elles s\'ecrivent
sous la forme~: $x(t) = a_0 + b_0t + c_0t^2 + d_0t^3$ et $y(t) = a_1 +
b_1t + c_1t^2 + d_1t^3$.

Une courbe de Bezier est d\'efinie par ses deux extremit\'es de
position respectifs $(x_0,y_0) = (x(0),y(0))$ et $(x_1,y_1) =
(x(1),y(1))$ et ses deux demi-tangentes $(u_0,v_0) = (x'(0), y'(0))$
et $(u_1,v_1) = (x'(1), y'(1))$. Voir le dessin. Pour calculer les
coefficients $a_0, b_0, c_0$ et $d_0$, il suffit de r\'esoudre le
syst\`eme suivant :
 
$$\left\{\begin{array}{ll}
  x(0)  & = b_0\\
  x'(0) & = a_0\\
  x(1)  & = a_0+b_0+c_0+d_0 \\
  x'(1) & = b_0+2c_0+3d_0 \\
\end{array} \right. 
\Longrightarrow 
\left\{\begin{array}{ll}
     a_0 & = x_0\\
     b_0 & = u_0\\
     c_0 & = 3x_1-3x_0-2u_0-u_1\\
     d_0 & = u_1-2x_1+2x_0+u_0\\
\end{array} \right. $$

On fait de m\^eme pour les coefficients de $y(t).$

\dessin{bez}{0.8}{Courbe de Bezier}{Bezier}

\subsection{R\'eduction du nombre d'objets dans le c\^one de vision}\label{frustum}
Du fait de la r\'egularit\'e de la forme de la ville, on peut
r\'eduire le nombre d'objets \`a consid\'erer dans la projection de
visualisation (frustum en anglais). Lors de cette projection, OpenGL
consid\`ere tous les objets de la sc\`ene. Il est donc dans ce cas
utile de refaire \`a la main la projection au lieu de laisser faire
OpenGL. On d\'efinit les objets que l'on veut projeter et on fait la
projection apr\`es avoir retrouv\'e les param\`etres de la projection
utilis\'es par OpenGL qui sont les deux matrices : {\tt
  GL\_PROJECTION\_MATRIX, GL\_MODELVIEW\_MATRIX}.

\subsection{Menu Delphi}

Lors de la deuxi\`eme soutenance, nous avons pr\'esent\'e le menu
r\'ealis\'e sous Delphi. La fen\^etre comporte trois onglets, dont
deux principaux.
\begin{itemize}
\item Dans le premier, une combobox permet de s\'electionner la
  voiture du joueur. Le coloris est alors affich\'e sous forme de
  photo (au format jpeg).
\item Dans le deuxi\`eme onglet, le joueur modifie les param\`etres du
  jeu tels que la r\'esolution de l'\'ecran, le nombre de bits, la
  densit\'e de voitures, la pluie, le jour ou la nuit, l'activation du
  son, le pourcentage du nombre de terrains par rapport au nombre de
  batiments et enfin la pente maximale des routes.
\item Le dernier onglet, sert aux cr\'edits. 
\end{itemize}

\dessin{menu}{0.4}{Le menu d'Ecstasy}{menu}

\subsection{Sons et video}

L'utilisation de DirectSound permet d'augmenter le r\'ealisme du jeu
en mettant un bruit de fond de ville.

\subsection{Installation}

L'utilisqtion du logiciel {\tt Setup2Go} a permis de cr\'eer le fichier
d'installation. Sur le cdrom, on trouve l'executable, les sources, les
dll, la page web et tous les rapports. L'installation ne pose aucun
probl\ `eme, il suffit de cliquer sur {\tt setup.exe}.

\section{Web et credits}
\subsection{Page web}
On peut t\'el\'echarger ce projet (en version lite et compl\`ete),
ainsi que ce rapport sur le site {\tt
  www.epita.fr/$\sim$quadra\_q}.

\subsection{Remerciments}

Quentin remercie :
\begin{itemize}
\item son p\`ere, pour lui avoir expliquer le principe de la moindre action, 
\item la communaut\'e web de Midtown Madness II, pour ses mod\`eles de
  voitures 3D librement t\'el\'echargeables.
\end{itemize}

\begin{thebibliography}{50}
\bibitem{STARDUST} http://www.soton.ac.uk/$\sim$trawww/stardust
\bibitem{ARNO} V. Arnold, \emph{M\'ethode math\'ematiques de la
    m\'ecanique classique}, Mir, Moscou, 1974.
\bibitem{APPEL} P. Appel,\emph{Trait\'e de m\'ecanique rationnelle},
  Gauthier-Villars, 1921.
\bibitem{OGL} M. Woo, J. Neider, T. Davis D. Shreiner \emph{OpenGL},
  Campus Press, 2000.
\bibitem{LM35} Xavier Michelon, xavier@linuxgraphic.org, \emph{OpenGL
    : G\'en\'eration de terrain}, Linux Magazine \No 33 --- 34.  
\bibitem{LM33} Xavier Michelon, xavier@linuxgraphic.org, \emph{OpenGL
    : G\'en\'eration de terrain}, Linux Magazine \No 33 --- 34.  
\bibitem{LM31} Xavier Michelon, xavier@linuxgraphic.org, \emph{OpenGL
    : Placage de textures}, Linux Magazine \No 31 --- 32.  
\bibitem{LM30} Xavier Michelon, xavier@linuxgraphic.org, \emph{OpenGL
    : Ecl\'erage et mat\'eriaux}, Linux Magazine \No 30.  
\bibitem{LM29} Xavier Michelon, xavier@linuxgraphic.org, \emph{OpenGL
    : Les transformations de visualisation}, Linux Magazine \No 29.  
\bibitem{LM28} Xavier Michelon, xavier@linuxgraphic.org, \emph{OpenGL
    : Transformations}, Linux Magazine \No 28.  
\bibitem{LM26} Xavier Michelon, xavier@linuxgraphic.org, \emph{Programmer avec OpenGL}, Linux Magazine \No 26 --- 27.

\end{thebibliography}
\end{document}
